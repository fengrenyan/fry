

\begin{abstract}
随着计算机系统越来越复杂,系统正确性和系统及其系统描述(规范,specification)之间的一致性越来越难以得到保证。
模型检测是一个保证系统正确性行之有效的方法之一。然而在模型检测中,若系统不满足给定的规范(即与规范不一致),如何更新系统使其能够与规范一致是长时间以来的一个重要问题。与这个问题密切相关的两个概念是最强必要条件(the strongest necessary condition, SNC)和最弱充分条件(the weakest sufficient condition, WSC),其分别对应于形式化验证中的最强后件(the strongest post-condition, SP)和最弱前件(the weakest precondition, WP)。此外,随着对系统信息越来越清晰,现有的规范不可避免地会与新的知识有冲突。
此时,如何将之前融入的元素在不影响其他信息的情况下“移除”也是个亟待解决的问题。

系统规范的描述语言以时序逻辑为主。其中$\CTL$(Computation tree logic)是一种重要的分支时间时序逻辑,其具有模型检测能多项式时间完成的特性,因此被广泛用于系统规范描述中。但是$\CTL$具有表达能力不够强的缺陷,$\mu$-演算($\mu$-calculus)是一种比$\CTL$表达能力更强但模型检测更加复杂的逻辑语言。因而,本文以这两种逻辑语言为研究背景,探索这两种语言下的上述问题的解决方法。

遗忘是一种知识抽取的技术,其被应用于信息隐藏、冲突解决和计算逻辑差等领域。本文从遗忘的角度出发解决上述提到的问题,
主要研究成果如下:

1. 给出$\CTL$下的遗忘的概念及其相关性质。首先,本文从模型在某个原子命题集合上互模拟的角度给出了遗忘的定义;其次,本文探讨了遗忘算子的代数属性,包括模块性、交换性和同质性;第三,表达定理表明遗忘和Zhang等人提出的四条公设具有当且仅当的关系,即:遗忘的结果满足四条准则,且满足那四条公设的公式为遗忘的结果。

2. 提出一种基于归结的方法计算$\CTL$下的遗忘。该方法使用Zhang等人提出的归结系统,在这个过程中需要将$\CTL$公式转换为$\CTLsnf$子句(separated normal form with global clauses for \CTL)的集合,最后再将具有索引的$\CTLsnf$子句转换为$\CTL$。在这一过程中需要计算遗忘的公式总是和各个过程的输出保持互模拟等价关系。

3. 给出了$\CTL$下遗忘封闭的情形——约束的$\CTL$。在这种情形下限制了公式的长度为$n$、公式所依赖的模型个数为有限个及构成公式的原子命题是有限的。此时,公式的模型可以用其特征公式——$\CTL$公式来表示。因此,遗忘的结果可以由其所有模型在给定原子命题集合上的特征公式的吸取来表示,显然该公式是一个$\CTL$公式(即:遗忘在这种情形下是封闭的)。

4. 研究了$\mu$-演算下的遗忘。$\mu$-演算是一种具有均匀插值(uniform  interpolation)性质,本文说明了$\mu$-演算下的遗忘与均匀插值是等价的,这意味着$\mu$-演算下的遗忘是封闭的,这是其与$\CTL$的不同。此外,研究了$\mu$-演算下遗忘的基本属性和复杂性,为均匀插值的研究提供了新的角度。

5. 给出了遗忘与WSC(SNC)和知识更新(knowledge update)的关系。WSC对模型的验证和修改具有重要作用,现有方法只能计算可终止模型的WSC,而像反应式系统这类不可终止的系统的WSC如何计算没有有效的方法。本文通过遗忘给出了计算WSC(SNC)的方法,并用遗忘定义了知识更新使得其满足Katsuno等人提出的知识更新应满足的八条公设。

6. 实现了2中提到的基于归结的计算计算$\CTL$下的遗忘的方法,并做了相应的实验。从标准数据集和随机产生的数据集里做了两组实验,分别为计算遗忘和SNC。实验表明公式越长或遗忘的原子个数越多,效率越低;此外,在随机产生的公式的大部分情况下能计算出SNC。

其意义主要为时序逻辑下的遗忘理论的研究提供了框架,并为模型更新提供了辅助工具——WSC。




\keywords{遗忘理论(forgetting),最强必要条件(SNC),最弱充分条件(WSC),知识更新(knowledge update)}
\end{abstract}


\begin{englishabstract}
 With the software (hardware) systems of a computer becoming more and more complex, it gets hard to guarantee the correctness of systems and the consistency between systems and its specification.
Model checking is a valid method to ensure the correctness of systems. However, it is an important problem in model checking to fix the system to make it consistent with its specification when the system does not satisfy the given specification.
Moreover, in such a scenario, two logical notions introduced by E. Dijkstra are highly informative: the \emph{strongest necessary condition} (SNC) and the \emph{weakest sufficient condition}  (WSC)  of a given specification. These correspond to the \emph{strongest post-condition} (SP) and the \emph{weakest precondition} (WP) of such specification, respectively.
Besides, the specification at hand is an unavoidable conflict with the new knowledge when the information of the system becomes clearer. In this case, another problem that needs to solve is to ``\emph{eliminate}” the containing elements without affecting the other information.

\emph{Forgetting}, a technique to distill knowledge, was used to hide information, solve the conflict, compute logic differences, and so on. This paper solves the above problems from the point of forgetting. The major contributions are as follows:

1. Given the definition and properties of forgetting in $\CTL$. First, this paper defines forgetting from the point of bisimulation over the given signature. Second, we explore the algebraic properties, including modularity, commutativity, and homogeneity, of forgetting. Third, the expression theorem shows that there is an ``if and only if” relation between forgetting and the forgetting postulates proposed by Zhang et al., i.e., the result of forgetting satisfies the forgetting postulates, and the formula which satisfies the forgetting postulates is the result of forgetting.

2. Proposed a resolution-based method to compute the forgetting in $\CTL$. This approach bases the resolution system proposed by Zhang et al., and the $\CTL$ formula is transformed into a set of $\CTLsnf$ clauses (separated normal form with global clauses for $\CTL$) at first. At the end of the approach, the $\CTLsnf$ clauses are transformed into a $\CTL$ formula to obtain the forgetting result. It is noteworthy that the output of each process is bisimilar equivalent of the input $\CTL$ formula.

3. We outline a situation in which the forgetting is closed. In this case, the length of formulas are limited to integer $n$, and the number of atoms for formulas and Kripke structures is finite. To prove that it is closed, we define the characterizing formula (i.e., a \CTL\ formula) of the (finite) initial \MPK-structure and show that each \CTL\ formula is equivalent to a disjunction of the characterizing formulas of its models.  This fact means that the result of forgetting some atoms from a \CTL\ formula always exists. 

4. Explored the forgetting in $\mu$-calculus. $\mu$-calculus is a kind of logic which have uniform interpolation. This paper shows that the uniform interpolation and forgetting in $\mu$-calculus are equivalent, this means that the forgetting in $\mu$-calculus is closed which is the biggest difference between $\mu$-calculus and $\CTL$. Moreover, the properties and complexity results related to forgetting are given, which proposes a new point for studying uniform interpolation.

5. Given the relation between forgetting and WSC (SNC or knowledge update). WSC is important to the verification and modification of the system. However, the existing methods can only compute the WSC of a terminable system, and the WSC can not be obtained in non-terminable systems (e.g., the reactive system). This paper shows how to compute the WSC of the reactive system and define the knowledge update using forgetting to satisfy the postulates proposed by~\citeauthor{katsuno91mendelzon}.

6. Implemented the algorithm proposed in 2, and some experiments are shown. Two experiments, computing forgetting, and SNC, were performed for standard and randomly generated datasets. Experiments show that the longer the formula is or the more atoms are forgotten, the lower the efficiency is. Moreover, SNC can be calculated in most cases of randomly generated formulas.

Its significance mainly provides a framework for the study of forgetting theory under temporal logic and provides an auxiliary tool for the model update.


\englishkeywords{forgetting,the strongest necessary condition,the weakest sufficient condition,knowledge update}
\end{englishabstract}
