\chapter{总结与展望}\label{chapter09}
{\em 本章首先总结文中针对$\CTL$和$\mu$-演算下的遗忘理论做出的研究工作,概括文中使用的方法及取得的研究成果。其次,讨论分析本文工作中存在的不足之处,并基于此对本文的后续研究内容进行了展望。}

\section{工作总结}
随着计算机系统日益变得复杂,描述系统规范的语言也变得越来越复杂。随着信息的更新,系统的规范也随着改变,因而急需有效的方法来提取相关原子命题下的知识。
遗忘是一种知识提取的方法,本文从语法和语义的角度探索了广泛应用于并行系统的$\CTL$下的遗忘。此外,也探索了表达能力更强的$\mu$-演算下的遗忘。
具体地,本文的主要工作总结如下:

\textcolor{blue}{按照1.3节的方式重新总结下面的内容。}

(1) 从语义的角度给出了$\CTL$和$\mu$-演算下的遗忘的定义,并探索了遗忘的基本性质。首先,给出了一般化的互模拟——在给定集合上的互模拟的定义。互模拟是描述两个系统行为的概念,若两个系统具有互模拟关系,则这两个系统具有相同的行为,即:在对应的状态上对相同的原子命题有相同的解释。公式刻画了其代表的模型的行为,从公式中遗忘掉给定的原子命题应该不影响该公式在其他原子命题在其模型上的行为表现,也即是新得到的公式的模型除了被遗忘的原子命题之外与原公式的模型有互模拟关系,即给定集合上的互模拟。基于此,遗忘的概念由给定集合上的互模拟给出。特别地,对于给定的原子命题集合$V$,若两个系统具有$V$-互模拟关系,则对于与$V$无关的公式,这两个系统同时满足或不满足该公式。其次,指出$\CTL$下的遗忘是不封闭的,而$\mu$-演算下的遗忘是封闭的,即存在$\CTL$公式的遗忘结果是不可用$\CTL$公式来表示的。
最后,探讨遗忘的代数属性,指出遗忘具有模块属性、交换属性和同质属性,这为计算遗忘提供了方便。

(2) 提出并实现了基于归结的$\CTL$下遗忘的计算方法。本文采用了Zhang等人的归结系统计算$\CTL$下的遗忘,并给出计算遗忘的算法。具体地,该算法以$\CTL$公式和原子命题集合$V$为输入,输出一个$\CTL$公式。该算法主要包括四个步骤:转换$\CTL$公式为$\CTLsnf$子句的集合、计算归结结果、移除包含$V$中元素的子句及将得到子句集合转换为$\CTL$公式。为此,本文给出了如何消除转换过程中引入的索引的方法,即移除掉索引后保持公式之间的互模拟等价。此外,为了消除转换过程中引入的新原子命题,提出了一般的Ackermann引理。

(3) 探索了约束情形下遗忘的封闭性。$\CTL$公式具有小模型性质,也即是对给定的$\CTL$公式,若该公式是可满足的,则其存在一个该公式大小指数的一个模型。因而本文讨论这种具有约束大小的公式,并讨论有限状态空间下的$\CTL$的遗忘。在这种情形下,$\CTL$公式的模型的个数是有限的,$\CTL$的遗忘是封闭的,且其遗忘的结果等于其所有模型的特征公式的吸取。此外,还给出了这种情形下计算遗忘的算法。尽管该算法从效率上来说是低效的,但是它是一种可靠且完备的方法。

(4) 使用遗忘计算SNC(WSC)和定义知识更新。对于给定的公式和原子命题集合,若遗忘掉除这些原子命题之外的原子命题的结果可用$\CTL$公式表示,则该结果一定是SNC(WSC),即:$\CTL$下可以用遗忘来计算SNC(WSC)。在约束情形下SNC(WSC)一定可以用遗忘方法来计算,因为这种情形下遗忘是封闭的。此外,当给定有限反应式系统的情形下,可以将该系统表示成特征公式,然后再使用遗忘来计算。最后,提出了两种定义知识更新的方法:基于遗忘的定义和基于模型的偏序关系的定义,并证明这两种方法定义的知识更新是等价的且满足Katsuno等人提出的八条基本准则。

(5) 实现了第\ref{chapter04}章提出的基于归结的算法,并做了实验,得到“遗忘的原子命题个数越少,计算效率越高”的结论。



\section{研究展望}
本文探讨了对系统设计至关重要的抽取信息的方法——遗忘,并使用该方法计算SNC(WSC)和定义知识更新。
文中指出$\CTL$下的遗忘不是封闭的,并提出了基于归结的计算遗忘的方法。
下面问题值得将来继续研究:

(1) 探索$\CTL$中遗忘封闭的子类。在系统规范描述中,有时候用到的公式不一定很复杂,也不一定需要用到所有的时序词。为此,探索简单并足够表达某些性质的$\CTL$公式的子类,在这些子类下遗忘是容易计算的。

(2) 探索如何使用计算出来的WSC(SNC)更新(修改)系统模型。特别地,当一个系统$\cal M$不满足规范$\phi$时,可以计算在某个命题集合$V$上的最弱充分条件$\psi$使得$\Hm$满足,即${\cal M}\models\psi\rto \phi$且$\psi$是$V$上的公式。这时,如何使用获得的WSC来更新系统得到新系统$\Hm'$使其满足$\phi$也是重要的。


(3) 尽管 $\CTL_{\ALL\FUTURE}$段下遗忘的模型检测和推理判定问题的复杂性已经给出,但是更多关于遗忘的问题的复杂性应该被探索:整个$\CTL$下的遗忘的模型检测和推理判定问题的复杂性。

(4) 研究算法~\ref{alg:compute:forgetting:by:Resolution}的完备性。尽管在第\ref{chapter04}中给出了算法正确性的分析,但是并没有讨论算法的完备性。在将来的工作中可以研究算法~\ref{alg:compute:forgetting:by:Resolution}的完备性,并探索其他可靠且完备的算法。

