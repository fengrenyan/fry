\begin{thanks}
	
此时此刻,真是难以表达自己内心的五味杂陈。

如果世界上有幸运儿,我想我应该是其中一个。作为贵州边远山村的孩子,能走到今天我是幸运的。而这些幸运都是由我身边的老师、亲人和朋友一点点累积的。

从小就听妈妈在耳边说:“你好好读书,只要你能读,我砸锅卖铁都会供你上学”。这句话看着简单,但却藏着深深的期许。妈妈小时候成绩极好,但是由于家庭贫寒和重男轻女的思想让她错过了上学的机会。因而将自己的希望寄托在我的身上,当然更多的是为了我的将来考虑,不想让我走她的老路。
对我来说,父母是人生道路上最初感染我的人。小时候每当看见爸爸妈妈辛辛苦苦的干活却过不上体面的生活,心里就暗暗发誓以后好好读书,将来让爸爸妈妈过上好一点的生活。
爸爸虽然不爱说话,但是心里也是及希望我考上大学的。
也是这些生活的点滴让我走到了今天,尽管途中很多困难,但都咬牙挺过。

老师,是我人生道路上指引方向、解决疑惑的人。从小学到现在,我的班主任或者导师在我的学习道路上都扮演着重要的角色。让我最记忆深刻的是高中班主任张焱老师及大学班导师谭熹微老师,他们是我在人生岔路口走向正途的引导人。而让我最受益的导师当属硕士及博士导师王以松老师,他不仅是我学术上的导师,也是我人生方向的引领者。王老师严谨的科研态度让我获益良多,从小我就是个粗心大意的人,现在能在知识表示与推理这个要求严谨的方向有点点成果王老师功不可没。从硕士走向博士,王老师也给了我很好的指向,并给了我出国留学的机会。说到留学,两位阿姆斯特丹自由大学的导师(Erman Acar 和 Stefan Schlobach)也是非常感谢的,他们给了我很多做科研的指导,尤其是Erman Acar对我的英文写作有很多帮助。此外,也感谢国防科技大学的刘万伟老师对我博士期间论文的指导。

人是群居动物,良好的群体环境不仅对生活大有裨益,对学术更加有帮助。感谢人工智能团队在我这七年(硕士+博士)的学习生涯中的帮助,从他们身上我学到了很多,他们的帮助让我感到了幸福和温暖。同时,感谢实验室的师兄弟姐妹在日常生活中给予我的帮助,读书期间的温馨和睦相处和茶余饭后时的谈笑风生都让我感受到家的温暖。



感谢贵州大学计算机学院的老师为本文工作所提供的支持和帮助;感谢秦永彬教授、彭长根教授、张明义教授等为本文的选题和研究方案的设计所提出的宝贵意见和建议。



最后,感谢参与该博士学位论文评审、答辩的诸位专家学者,感谢您们为提高我的博士学位论文质量所提出的宝贵修改意见和建议。
	
	
\end{thanks}
