

\begin{abstract}
随着计算机系统越来越复杂,系统正确性和系统及其系统描述(规范,specification)之间的一致性越来越难以得到保证。
模型检测是一个保证系统正确性行之有效的方法之一。然而在模型检测中,若系统不满足给定的规范(即与规范不一致),如何更新系统使其能够与规范一致是长时间以来的一个重要问题。与这个问题密切相关的两个概念是最强必要条件(the strongest necessary condition, SNC)和最弱充分条件(the weakest sufficient condition, WSC),其分别对应于形式化验证中的最强后件(the strongest post-condition, SP)和最弱前件(the weakest precondition, WP)。此外,随着对系统信息越来越清晰,现有的规范不可避免地会与新的知识有冲突。
此时,如何将之前融入的元素在不影响其他信息的情况下“移除”也是个亟待解决的问题。

系统规范的描述语言以时序逻辑为主。其中$\CTL$(Computation tree logic)是一种重要的分支时间时序逻辑,其具有模型检测能多项式时间完成的特性,因此被广泛用于系统规范描述中。但是$\CTL$具有表达能力不够强的缺陷,$\mu$-演算($\mu$-calculus)是一种比$\CTL$表达能力更强但模型检测更加复杂的逻辑语言。因而,本文以这两种逻辑语言为研究背景,探索这两种语言下的上述问题的解决方法。

遗忘是一种知识抽取的技术,其被应用于信息隐藏、冲突解决和计算逻辑差等领域。本文从遗忘的角度出发解决上述提到的问题,
主要研究成果如下:

1. 给出$\CTL$下的遗忘的概念及其相关性质。首先,本文从模型在某个原子命题集合上互模拟的角度给出了遗忘的定义;其次,本文探讨了遗忘算子的代数属性,包括模块性、交换性和同质性;第三,表达定理表明遗忘和Zhang等人提出的四条准则具有当且仅当的关系,即:遗忘的结果满足四条准则,且满足那四条公设的公式为遗忘的结果。

2. 提出一种基于归结的方法计算$\CTL$下的遗忘。该方法使用Zhang等人提出的归结系统,在这个过程中需要将$\CTL$公式转换为$\CTLsnf$子句(separated normal form with global clauses for \CTL)的集合,最后再将具有索引的$\CTLsnf$子句转换为$\CTL$。在这一过程中需要计算遗忘的公式总是和各个过程的输出保持互模拟等价关系。

3. 给出了$\CTL$下遗忘封闭的情形——约束的$\CTL$。在这种情形下限制了公式的长度为$n$、公式所依赖的模型个数为有限个及构成公式的原子命题是有限的。此时,公式的模型可以用其特征公式——$\CTL$公式来表示。因此,遗忘的结果可以由其所有模型在给定原子命题集合上的特征公式的吸取来表示,显然该公式是一个$\CTL$公式(即:遗忘在这种情形下是封闭的)。

4. 研究了$\mu$-演算下的遗忘。$\mu$-演算是一种具有均匀插值(uniform  interpolation)性质,本文说明了$\mu$-演算下的遗忘与均匀插值是等价的,这意味着$\mu$-演算下的遗忘是封闭的,这是其与$\CTL$的不同。此外,研究了$\mu$-演算下遗忘的基本属性和复杂性,为均匀插值的研究提供了新的角度。

5. 给出了遗忘与WSC(SNC)和知识更新(Knowledge update)的关系。WSC对模型的验证和修改具有重要作用,现有方法只能计算可终止模型的WSC,而像反应式系统这类不可终止的系统的WSC如何计算没有有效的方法。本文通过遗忘的方法给出了计算WSC(SNC),并用遗忘定义了知识更新使得其满足。。等人提出的知识更新应满足的八条公设。

6. 实现了2中提到的基于归结的计算计算$\CTL$下的遗忘的方法,并做了相应的实验。从标准数据集和随机产生的数据集里做了两组实验,分别为计算遗忘和SNC。实验表明公式越长或遗忘的原子个数越多,效率越低;此外,在随机产生的公式的大部分情况下能计算出SNC。

其意义主要为时序逻辑下的遗忘理论的研究提供了框架,并为模型更新提供了辅助工具——WSC。




\keywords{遗忘理论(forgetting),最强必要条件(SNC),最弱充分条件(WSC),知识更新(knowledge update)}
\end{abstract}


\begin{englishabstract}
 The challenges of privacy and security caused by the rapid development of informatization and in-depth applications, have become a bottleneck restricting data opening, sharing, exchange, and application, and have attracted great attention from the legal and academic communities. From the perspective of technology, the differential privacy (DP) protection algorithm, as an important privacy protection technology, is not mature enough in the research of data privacy protection for multi-dimensional and complex associations. Firstly, due to the mixed data types, sparseness, and large domain value space etc, the multi-dimensional data processing of DP is faced with the challenges such as privacy vulnerability and low computational efficiency. Secondly, the relevance of data fusion, background knowledge attacks and strategic adversary attacks, and the contradiction between data privacy and usability have become prominent issues. For the problems mentioned above, it is a better solution to investigate the trade-off and optimization of privacy and utility from the perspective of the game theory. Thus, this article mainly focuses on the crucial problem of the trade-off between privacy and utility. Based on information entropy, optimization theory and game equilibrium and other related theories and methods, the equilibrium and optimization models are constructed as the main line of this research. A series of results have been achieved in designing of privacy quantification methods, constructing and solving game model between privacy and utility, optimization model establishment and solving, etc., which provide a reference for solving privacy protection issues from the perspective of combining technology and management. The major contributions can be summarized as follows.

1. The information entropy metric models and methods of DP are proposed. For the quantitative problem of privacy, the noisy DP communication model and formalization statement are defined based on the Shannon's fundamental communication model and the randomized perturbation principle of DP. Further, the notions of information entropy, conditional entropy, joint entropy, mutual information and conditional mutual information, etc., are defined under the differential privacy model, and then, the privacy metric models with information entropy as the core are designed. For the problem of multi-dimensional and correlated attributes, based on the graph and Markov model, etc., a privacy metric model and method for multi-dimensional and correlated attributes is proposed. Then, the upper and lower bounds of privacy leakage are quantified by using data processing inequality and Fano's inequality. Theoretic analysis and experimental results are demonstrating the proposed metric model and method can effectively achieve the goal of DP quantification, and further provide basic support for privacy leakage risk assessment and privacy protection mechanism design.


2. The differential privacy optimization model with background knowledge attacks is proposed. Based on the established fundamental communication model of the DP, lossy compression theory and the proposed privacy metric model, the adversary model which has relevant background knowledge is established, and further the DP communication model with background knowledge attacks is proposed. By using conditional mutual information measures privacy, this paper updates the form of the well-known rate distortion function, and proposes the differential privacy optimization model with background knowledge attacks. Further, the alternating minimization iteration algorithm solving the proposed optimization model is designed and implemented based on the Blahut-Arimoto alternating minimization method, and the computation complexity analysis is provided. Theoretic analysis and experimental results are demonstrating the proposed method have significant advantages in data quality and privacy leakage when compared with the existing symmetrical channel mechanism.


3. The orderly randomized response perturbation (ORRP) scheme is proposed. For the problem of low efficiency and privacy vulnerability when deal with multi-dimensional data using local differential privacy, and facing the privacy protection requirements of data collection scenarios, this paper proposes an orderly randomized response perturbation scheme. The proposed ORRP scheme effectively solves the impact of the existing privacy protection mechanisms ignoring data distribution, and the problem of low computing efficiency caused by the large processing domain value space and sparse data. To be specific, the proposed ORRP scheme based on the prior proposed privacy metric model. A mutual information optimization model subjects to a given data quality loss constraint to minimize privacy leakage, is proposed by analyzing and quantifying the requirements of privacy and data quality. Further, the probability density function (PDF) of the optimal privacy mechanism is computed by the means above, and it is used to achieve randomized perturbation. Meanwhile, referring to the independent parallel channel model, the above methods are extended to the case of multi-dimensional data. Finally, theoretical analysis and experimental simulations are given in terms of privacy leakage, data usability quality, and correlation loss. The results demonstrate that the proposed ORRP has more advantages than the existing methods in terms of data semantic integrity, privacy and data availability quality.


4. The privacy-preserving attack and defense (PPAD) game model is proposed. For the problem of informed and strategic adversary in the differential privacy system, the selection strategy of differential privacy protection is designed around the data collection scenarios. On the basis of the above, the PPAD game model is proposed, and the trade-off between privacy and utility in the protection of differential privacy is achieved by solving the equilibrium. The proposed scheme is based on the established differential privacy basic communication model. The privacy minimax optimization model is established by analyzing the privacy goals of defender and strategic attacker, and further the formalization statement of PPAD is provided, which includes players' sets, strategic spaces and payoff functions etc. This paper cleverly uses the connotation and extension of private mutual information to construct the utility function of privacy protection, and finally realized the construction of a two-person zero-sum (TPZS) game model. Then, this paper provides the game analysis by using von Neumann's minimax theorem and concave-convex game, and further designs a strategy optimization selection algorithm to calculate saddle point based on the optimal strategy response. Theoretic analysis and numeric simulation results show that the proposed model and method can effectively solve the problem of comparison between equivalent privacy mechanisms, and also can be used for privacy leakage risk assessment in the worst case of privacy protection.







\englishkeywords{Privacy metric, differential privacy protection, rate-distortion function, game equilibrium, optimization model
}
\end{englishabstract}
