\chapter{总结与展望}\label{chapter09}
{\em 本章首先总结文中针对$\CTL$和$\mu$-演算下的遗忘理论做出的研究工作,概括文中使用的方法及取得的研究成果。其次,讨论分析本文工作中存在的不足之处,并基于此对本文的后续研究内容进行了展望。}

\section{工作总结}
随着计算机系统日益变得复杂,描述系统规范的语言也变得越来越复杂。随着信息的更新,系统的规范也随着改变,因而急需有效的方法来提取相关原子命题下的知识。
遗忘是一种知识提取的方法,本文从语法和语义的角度探索了广泛应用于并行系统的$\CTL$下的遗忘。此外,也探索了表达能力更强的$\mu$-演算下的遗忘。
具体地,本文的主要工作总结如下:

(1) 从语义的角度给出了$\CTL$和$\mu$-演算下的遗忘的定义,并探索了遗忘的基本性质。首先,给出了一般化的互模拟——在给定集合上的互模拟的定义。互模拟是描述两个系统行为的概念,若两个系统具有互模拟关系,则这两个系统的具有相同的行为,即:在对应的状态上对相同的原子命题有相同的解释。公式刻画了其代表的模型的行为,从公式中遗忘掉给定的原子命题应该不影响该公式在其他原子命题在其模型上的行为表现,也即是新得到的公式的模型除了被遗忘的原子命题之外与原公式的模型有互模拟关系,即给定集合上的互模拟。基于此,遗忘的概念由给定集合上的互模拟给出。特别地,对于给定的原子命题集合$V$,若两个系统具有$v$-互模拟关系,则对于与$V$无关的公式,这两个系统同时满足或不满足该公式。其次,指出$\CTL$下的遗忘是不封闭的,而$\mu$-演算下的遗忘是封闭的,即存在$\CTL$公式的遗忘结果是不可用$\CTL$公式来表示的。
最后,探讨遗忘的代数属性,指出遗忘具有模块属性、交换属性和通知属性,这为计算遗忘提供了方便。

(2) 提出并实现了基于归结的$\CTL$下遗忘的计算方法。本文采用了\citeauthor{zhang2014resolution}的归结系统计算$\CTL$下的遗忘,并给出计算遗忘的算法。具体地,该算法以$\CTL$公式和原子命题集合$V$为输入,输出一个$\CTL$公式。该算法主要包括四个步骤:转换$\CTL$公式为$\CTLsnf$子句的集合、计算归结结果、移除包含$V$中元素的子句及将得到子句集合转换为$\CTL$公式。为此,本文给出了如何消除转换过程中引入的索引的方法,即移除掉索引后保持公式之间的互模拟等价。此外,为了消除转换过程中引入的新原子命题,提出了一般的Ackermann引理。

(3) 探索了约束情形下遗忘的封闭性。$\CTL$公式具有小模型性质,也即是对给定的$\CTL$公式,若该公式是可满足的,则其存在一个该公式大小指数的一个模型。因而本文讨论这种具有约束大小的公式,并讨论有限状态空间下的$\CTL$的遗忘。在这种情形下,$\CTL$公式的模型的个数是有限的,$\CTL$的遗忘是封闭的,且其遗忘的结果等于其所有模型的特征公式的吸取。此外,还给出了这种情形下计算遗忘的算法。尽管该算法从效率上来说是低效的,但是它是一种可靠且完备的方法。

(4) 使用遗忘计算SNC(WSC)和定义知识更新。对于给定的公式和原子命题集合,若遗忘掉除这些原子命题之外的原子命题的结果可用$\CTL$公式表示,则该结果一定是SNC(WSC),即:$\CTL$下可以用遗忘来计算SNC(WSC)。在约束情形下SNC(WSC)一定可以用遗忘方法来计算,因为这种情形下遗忘是封闭的。此外,当给定有限反应式系统的情形下,可以将该系统表示成特征公式,然后在使用遗忘来计算。最后,提出了两种定义知识更新的方法:基于遗忘的定义和基于模型的偏序关系的定义,并证明这两种方法定义的只是更新是等价的且满足~\citeauthor{katsuno91mendelzon}提出的八条基本准则。




\section{研究展望}
本文探讨了对系统设计至关重要的抽取信息的方法——遗忘,并使用该方法计算SNC(WSC)和定义知识更新。
文中指出$\CTL$下的遗忘不是封闭的,并提出了基于归结的计算遗忘的方法。
但是仍然存在一些问题没有且将来可以做的问题如下:

(1) 探索$\CTL$中遗忘封闭的子类。在系统规范描述中,有时候用到的公式不一定很复杂,也不一定需要用到所有的时序词。为此,探索简单并足够表达某些性质的$\CTL$公示的子类,在这些子类下遗忘是容易计算的。

(2) 探索如何使用计算出来的WSC(SNC)更新(修改)系统模型。特别地,当一个系统$\cal M$不满足规范$\phi$是,

There remain some interesting directions that deserve further investigation: First, although the initial results for the  complexity analysis of some problems (including model checking and  Entailment of forgetting) for the $\CTL_{\ALL\FUTURE}$ fragment was reported in our earlier work  KR~\cite{renyansfirstpaper}, a thorough investigation is still needed for the general case (namely, \CTL). Second, we did not yet resolve whether the resolution-based method is complete for forgetting. Second, we still do not explore how can the WSC obtained by our algorithm be used to update the original model system.
More specifically, when a transition system $\cal M$ does not satisfy a specification $\phi$, one can evaluate the weakest sufficient condition  $\psi$ over a signature $V$ under which ${\cal M}$ satisfies $\phi$, viz., ${\cal M}\models\psi\rto \phi$ and $\psi$ mentions only atoms from $V$. It is worthwhile to explore how the condition $\psi$ can guide the design of a new transition system ${\cal M}'$ satisfying $\phi$.
Last but not least, different fragments, in which the result of forgetting always exists, is of central interest for the future research avenue.


当前,差分隐私在数据隐私保护中发挥重要的作用,应用范围涉及数据发布、数据收集、数据分析、机器学习等领域,对其应用的研究仍需要积极的推进。虽然本文基于信息论和对策博弈论的基础理论方法,从均衡优化的角度做出了一些有意义的探索工作,但是本文的研究中尚存在一些值得深入研究的问题。具体包括有:

(1) 在隐私度量方面,研究表达用户隐私敏感偏好强度的度量方法,建立差分隐私$\epsilon$-度量、用户个性化隐私需求和信息熵度量的联系,为个性化的差分隐私研究奠定基础。进一步,在面向多维关联数据情景研究并提出个性化的差分隐私方案是一个值得深入研究的重要方向。

(2) 在权衡隐私与效用的优化模型研究方面,研究最大化数据效用的优化模型,并设计具体的隐私机制是非常有价值的方向。其次,基于所建立的差分隐私通信模型,从信道容量的角度考虑差分隐私的最大信息传输率,对差分隐私保护系统中隐私信息率的定量化研究也是一个值得探索的方向。


(3) 在隐私与效用的均衡优化方面,基于博弈均衡理论的指导,利用非完全信息的动态博弈、静态博弈,建立两方或多方的攻防博弈模型探讨差分隐私保护的最优策略问题仍然是值得研究的方向。此外,基于量化信息流思想,构建差分隐私的信息泄露博弈仍然是值得关注的研究点。

%本文得到了国家自然科学基金项目和贵州省研究生科研基金项目的资助,现已发表《计算机学报》、《电子学报》等SCI/EI学术期刊论文3篇。此外,文中的部分内容已投稿至IEEE Transactions on Knowledge and Data Engineering(TKDE)期刊,目前论文大修后复审中(Under Review)。由于作者学识水平有限,下一步工作对上述展望的三个方面继续开展深入研究。

