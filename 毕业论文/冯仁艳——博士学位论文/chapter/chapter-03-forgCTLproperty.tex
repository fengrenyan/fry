
\chapter{CTL下的遗忘理论的定义及其语义属性}\label{chapter03}
{\em 本章首先通过扩展互模拟的概念,给出$\CTL$下遗忘理论的定义。其次,探索遗忘理论的一般通用属性,这些属性包括:模块化(Modularity)性质、交换性(Commutativity)、齐次性(Commutativity)等属性。}

\section{引言}
从一个公式中“遗忘”掉一些原子命题得到的结果应该不违背定义在剩余原子命题集合上的公式,也就是说对于剩余原子命题集合上的公式,原始公式能够逻辑蕴涵它当且仅当遗忘得到的结果能过逻辑蕴涵它。从模型的角度来讲,遗忘得到的结果的模型与原始公式的模型在除去被遗忘的那些原子命题之后是能够想互模拟的。\emph{互模拟}描述的是两个在行为上能够相互替代的转换系统\cite{Baier:PMC:2008}。在本文中,转换系统被描述成为Kripke结构。因此为了描述遗忘理论,这部分给出在给定原子命题集合上的Kripke结构(或$\Ind$-结构)上的互模拟的定义及其性质。

基于互模拟的概念,给出了$\CTL$下遗忘理论的定义。与后面章节将要讲述的约束$\CTL$下的遗忘相对应,这部分探索没有约束的遗忘理论的一般属性。

\section{$V$-互模拟}

这部分给出定义在给定原子命题集合$V$上的互模拟的概念,本文称之为$V$-互模拟。尽管在文章~\cite{Yan:AIJ:2009}中给出了相似的概念,但是如在基础知识部分所述,$S5$的语义是定义在一种特殊的Kripke结构($\MPK$-解释)下的,因而不具有一般性。接下来探讨一种更加一般的$V$-互模拟。

为此,首先给出能够描述一定深度$n\in \mathbb{N}$的计算树之间的$V$-互模拟关系,记为$\Hb_n^V$。令$V\subseteq \Ha$是原子命题的集合,$i\in \{1,2\}$,$\Hm_i=(S_i,R_i,L_i,s_0^i)$(或$\Hm_i=(S_i,R_i,L_i,[\_]_i,s_0^i)$)是初始结构($\Ind$-Kripke结构),${\cal K}_i=(\Hm_i, s_i)$是$\MPK$-结构(或$\Ind$-结构)。$\Hb_n^V$被递归定义如下:
\begin{itemize}
	\item 若$L_1(s_1)-V=L_2(s_2)$,则$({\cal K}_1,{\cal K}_2) \in \Hb_0^V$;
	\item 对任意$n\ge 0$,若满足下面几个条件,则$({\cal K_1},{\cal K_2})\in \Hb_{n+1}^V$成立:
		\begin{itemize}
			\item $({\cal K}_1,{\cal K}_2) \in \Hb_0^V$;
			\item 对任意$(s_1,s_1')\in R_1$,存在$(s_2,s_2')\in R_2$使得$({\cal K}_1',{\cal K}_2') \in \Hb_n^V$;
			\item 对任意$(s_2,s_2')\in R_2$,存在$(s_1,s_1')\in R_1$使得$({\cal K}_1',{\cal K}_2') \in \Hb_n^V$。
		\end{itemize}
\end{itemize}
其中${\cal K}_i'=(\Hm_i,s_i')$。

当所谈及的原子命题的集合$V$很显然的时候,上述$\Hb_n^V$中的$V$可以省略,记为$\Hb_n$。
此时,$V$-互模拟关系就可以定义如下:
\begin{definition}[$V$-互模拟]\label{def:V-bisimulation}
	令$V$是$\Ha$的一个子集,$i\in \{1,2\}$, ${\cal K}_1$和${\cal K}_2$是$\MPK$-结构(或$\Ind$-结构)。
	\begin{itemize}
		\item ${\cal K}_1$和${\cal K}_2$是$V$-互模拟的,当且仅当对所有的$n \ge 0$都有$({\cal K}_1, {\cal K}_2)\in \Hb_n$。若${\cal K}_1$和${\cal K}_2$是$V$-互模拟的,则记为${\cal K}_1 \lrto_V {\cal K}_2$。
		\item 对$\Hm_i$上的路径$\pi_i=(s_{i,1},s_{i,2},\dots)$,若对于任意的$j\in \mathbb{N}_{\ge 1}$都有${\cal K}_{1,j} \lrto {\cal K}_{2,j}$,则$\pi_1 \lrto_V \pi_2$。其中${\cal K}_{i, j}=(\Hm_i, s_{i,j})$。
	\end{itemize}
\end{definition}

上述$V$-互模拟的定义是现有互模拟定义的一般化,这可以从下面几个方面来体现\footnote{在其他领域也有类似的定义,如:定义在数据库相关文献中的概念$k$-互模拟\cite{kaushik2002updates}。$k$-互模拟概念中涉及与本文$\Hb_n$类似的定义,只是其关系是从相反的方向(即:从孩子到父节点的方向)来说明的。此外,值得一提的是,本文的$V$-互模拟的概念是定义在$\MPK$-结构上的。}。
首先,当给定的$V$为空集且谈论指定的初始状态时,本文的$V$-互模拟与定义在Baier等文章里的互模拟等价(定义7.1\cite{Baier:PMC:2008})的概念一致。
其次,在同一文章里的基于状态的互模拟(定义7.7\cite{Baier:PMC:2008})是定义在给定结构的状态上的,因此与本文的$V$-互模拟(定义在结构的集合上)也不同。
最后,本文的$\Hb_n$的定义与Browne的论文中的状态等价$E_n$类似,只是后者是定义在状态上\cite{browne1988characterizing}而本文的定义在$\MPK$-结构(或$\Ind$-结构)上。


下面例子呈现结构之间的$V$-互模拟。
\begin{example}
	
\end{example}


 

\section{本章小结}
本章基于Shannon基本通信模型,给出差分隐私通信模型及其信道数学模型。以此通信模型为基本出发点,构建了一种属性关联、元组记录独立同分布的离散无记忆信源的$n$次扩展信源。在此基础上,引入信息熵、条件熵、互信息量、失真函数等概念给出了隐私信源熵、隐私熵与隐私泄露风险、数据效用的基本度量模型及方法。进一步,针对多维属性关联的隐私度量问题,基于信息熵提出了面向关联属性的差分隐私信息熵度量模型及方法。首先,利用互信息相关性分析方法解决关联属性相关度量化问题。随后,基于相关度给出关联属性的图表示。进一步,基于差分隐私的信道模型、属性关联图中隐私泄露关键路径,借助马尔可夫理论分析了关联属性导致敏感信息的隐私泄露量。最后,以真实数据集上的实验给出所提出的信息熵度量模型及方法的量化过程及结果。本章是研究差分隐私信息论度量的基础工作,为后续基于隐私信息度量研究权衡隐私与效用均衡优化模型以及隐私机制设计奠定基础。

