%声明文档类型和比例
\documentclass[aspectratio=169, 10pt, utf8, mathserif]{beamer}
%调用相关的宏包
\usepackage{ctex}
\usepackage{amsmath, amsfonts}
\usepackage{graphicx}
\usepackage{multicol} %分栏
\usepackage{booktabs} %表格功能包
\usepackage{multirow} %合并多行表格
\usepackage{enumerate} %有序编号
\usepackage{listings} %代码包
\usepackage{xcolor} %代码高亮包
\lstset{
	language=Matlab, %代码语言使用的是matlab
	frame=shadowbox, %把代码用带有阴影的框圈起来
	rulesepcolor=\color{red!20!green!20!blue!20}, %代码块边框为淡青色
	keywordstyle=\color{blue}\bfseries, %代码关键字的颜色为蓝色,粗体
	commentstyle=\color{red}\textit, %设置代码注释的颜色
	showstringspaces=false, %不显示代码字符串中间的空格标记
	numbers=left, %显示行号
	numberstyle=\tiny, %行号字体
	stringstyle=\ttfamily, %代码字符串的特殊格式
	breaklines=true, %过长的代码自动换行
	extendedchars=false,  %解决代码跨页时,章节标题,页眉等汉字不显示的问题
	escapebegin=\begin{CJK}{GBK}{hei},escapeend=\end{CJK} %防止中文报错
	texcl=true}

\usetheme{Antibes} %主题包之一,直接换名字即可
\usecolortheme{beaver} %主题色之一,直接换名字即可。
\usefonttheme{professionalfonts}

% 设置用acrobat打开就会全屏显示
\hypersetup{pdfpagemode=FullScreen}

% 设置logo
\pgfdeclareimage[height=2cm, width=2cm]{university-logo}{120701101}
\logo{\pgfuseimage{university-logo}}

%--------------正文开始---------------
\begin{document}
	
	%每个章节都有小目录
	\AtBeginSection[]
	{
		\begin{frame}<beamer>
			\tableofcontents[currentsection]
		\end{frame}
	}
	
	\title{Beamer 入门}
	\subtitle{利用已有主题实现自己的主题}
	\author[120701101]{120701101 \\ \small \href{mailto:183xxxxx30@163.com}{183xxxxxx530@.com}}
	\institute[公众号:120701101]
	{
		科学计算 \\
		MATLLAB和Python
	}
	\date{\today}
	\begin{frame}
		%\maketitle
		\titlepage
	\end{frame}
	
	\begin{frame}
		\frametitle{目录}
		\tableofcontents[hideallsubsections]
	\end{frame}
	
	\section{使用已有主题的方法}
	\subsection{主题样式颜色}
	\begin{frame}
		\frametitle{使用已有主题的方法}
		可以直接点击该链接\underline{\href{https://mpetroff.net/files/beamer-theme-matrix/}{已有的主题样式和主题颜色}}。
		横栏表示主题颜色,纵栏表示主题样式。\\
		将想套用的主题样式和颜色放到usetheme{Szeged}和usecolortheme{beaver}中即可。
	\end{frame}
	
	\section{公式及编号}
	\subsection{带编号的公式}
	\begin{frame}
		\frametitle{带编号的公式}
		现在展示一个带编号的公式:
		\begin{equation}
			f(x) = \frac{\mathrm e^{2x}}{\sin x}
		\end{equation}
	\end{frame}
	
	\subsection{不带编号的公式}
	\begin{frame}
		\frametitle{不带编号的公式}
		另外再展示一个不带编号的公式。
		\[
		\mathrm e^{\mathrm i \pi} + 1 = 0 
		\]
	\end{frame}
	
	\subsection{行内公式}
	\begin{frame}
		\frametitle{行内公式}
		以及一个行内公式$a^2 + b^2 = c^2$.
	\end{frame}
	
	\section{列表环境}
	\subsection{无序列表和逐条展示的功能}
	\begin{frame}
		\frametitle{列表}
		这是无序列表的样式,及逐条展示的功能。
		\begin{itemize}
			\item 无序列表标号1
			\pause
			\item 无序列表标号2
		\end{itemize}
	\end{frame}
	
	\subsection{有序列表}
	\begin{frame}
		\frametitle{有序列表}
		这是有序列表的样式及一次性的逐条展示功能。
		\begin{enumerate}[<+-|alert@+>]
			\item 这是1
			\item 这是2
		\end{enumerate}
	\end{frame}
	
	\section{块环境}
	\subsection{放某些特定的句子和公式}
	\begin{frame}
		\frametitle{块环境}
		\begin{block}{Beamer介绍}
			Beamer是\LaTeX 的一个文档类,主要用于学术报告幻灯片的制作,优点是跨平台性好,支持Windows,Mac等。导出的格式就是PDF。
		\end{block}
		\begin{block}{Beamer介绍}
			\begin{equation}
				\left \{
				\begin{aligned}
					f(x) &= 2x + b \\
					g(x) &= x + 9
				\end{aligned}	
				\right.
			\end{equation}
		\end{block}
		
	\end{frame}
	
	\section{图文并排}
	\begin{frame}
		\frametitle{左图右文字}
		\begin{columns}
			\column{.3\textwidth}
			\begin{figure}
				\centering
				\includegraphics[height=3cm, width=3cm]{120701101}
			\end{figure}
			\column{.7\textwidth}
			\begin{itemize}
				\item 公众号:120701101的Logo。
				\item 我利用矩阵的形状来模拟这些我喜欢的数字组合。
				\item 因为是非矢量图,所以放大后有损。
			\end{itemize}
		\end{columns}
	\end{frame}
	
	\section{代码环境}
	\begin{frame}[fragile] %must using [fragile]
		\frametitle{MATLAB代码}  
		\begin{lstlisting}[numbers=left, firstnumber=753]
			% 绘制图形
			x = 1 : 0.01 : 5;
			y = sin(x);
			plot(x, y)
		\end{lstlisting}
	\end{frame}
	
	\begin{frame}
		\zihao{-4}\centering{坚持学习,不是为了输赢。}
	\end{frame}
\end{document}