% This file is modified by Frans Oliehoek <faolieho@science.uva.nl>
% on 2005/12/18. (original file name: conference-ornate-20min.en.tex)

\documentclass[11pt, CJK]{beamer}

% This file is a solution template for:
% - Talk at a conference/colloquium.
% - Talk length is about 20min.
% - Style is ornate.



% Copyright 2004 by Till Tantau <tantau@users.sourceforge.net>.
%
% In principle, this file can be redistributed and/or modified under
% the terms of the GNU Public License, version 2.
%
% However, this file is supposed to be a template to be modified
% for your own needs. For this reason, if you use this file as a
% template and not specifically distribute it as part of a another
% package/program, I grant the extra permission to freely copy and
% modify this file as you see fit and even to delete this copyright
% notice. 

\mode<presentation>
{

	\usetheme{IAS_sidebarNav} 	 
	\setbeamercovered{transparent}

}


% to remove the navigation symbols:
%\setbeamertemplate{navigation symbols}{}

\usepackage[english]{babel}
% or whatever

\usepackage[utf8]{inputenc}
% or whatever

\usepackage{times}
\usepackage[T1]{fontenc}
% Or whatever. Note that the encoding and the font should match. If T1
% does not look nice, try deleting the line with the fontenc.

\usepackage{booktabs, multirow, enumerate}
\usepackage{ctex}
\usepackage{setspace}
% Delete this, if you do not want the table of contents to pop up at
% the beginning of each subsection:
\AtBeginSection[]
{
	\begin{frame}<beamer>
		\frametitle{目录}
		\tableofcontents[currentsection]
	\end{frame}
}
% If you wish to uncover everything in a step-wise fashion, uncomment
% the following command: 

\beamerdefaultoverlayspecification{<+->}
\newcommand{\tabincell}[2]{\begin{tabular}{@{}#1@{}}#2\end{tabular}}  %表格自动换行
% If you have a file called "university-logo-filename.xxx", where xxx
% is a graphic format that can be processed by latex or pdflatex,
% resp., then you can add a logo as follows:
% \pgfdeclareimage[height=0.5cm]{university-logo}{university-logo-filename}
% \logo{\pgfuseimage{university-logo}}
\graphicspath{{figures/}}
%\pgfdeclaremask{mymask}{figures/thu_logo2-mask}
%\pgfdeclareimage[width=0.45\textwidth,mask=mymask]{image2}{pictures/wai1}
\pgfdeclareimage[width=1.6cm,height=1.0cm]{institution-logo}{figures/gzulogo}
\logo{\pgfuseimage{institution-logo}}

%\includeonlyframes{mine}
\begin{document}
	\title[硕士中期检查答辩]{\CJKfamily{li}{基于蒙特卡洛树搜索的‘斗地主’研究}}
	
	\author[彭啟文]{(中期检查答辩报告){\vskip 23pt}指导教师:王以松~教授{\vskip 5pt}~~~~~答辩学生:彭啟文~~~~~~~~~~~~~~{\vskip 5pt}学~~~~号~:2017021834~~~}{\vskip 5pt}
	\date{2019-12-9}
	%\institute{{\vskip 15pt}\small 贵州大学计算机学院人工智能实验室}
	
	%\date{{\vskip -9pt}二{\fontsize{9pt}{9pt}\selectfont 〇一}九年十二月}
	
	\begin{frame}
		\titlepage
	\end{frame}
	
	\section{研究方案}
		\subsection*{研究内容}
			\begin{frame}
				\frametitle{~~研究内容}
				\begin{itemize}
					\item 研究内容:
					\begin{itemize}
						\setlength{\baselineskip}{20pt}
						\item
						本课题主要以国内比较流行的游戏‘斗地主’作为研究对象,蒙特卡洛树搜索为主要研究手段(辅以机器学习的方法),针对‘斗地主’游戏的特点设计算法。通过不断的实验方法改进,最终目标是该智能体能和人类进行正常的比赛并取得较好的效果
					\end{itemize}
				\end{itemize}
			\end{frame}
		\subsection*{研究方案}
			\begin{frame}
				\frametitle{~~研究方案}
				\begin{itemize}
					%\setlength{\parsep}{15pt}
					\setlength{\baselineskip}{16pt}
					\item 博弈开始阶段,按照‘斗地主’游戏规则,对该博弈者手牌进行拆分,使用贪心思想选择其中拆分手牌数较少的几种情况进行分析,以使所选拆分情况比较合理
					\vskip 8pt
					\item 根据剩余手牌、玩家已出牌、每个玩家的手牌张数以及结合扑克特征等信息,进行博弈树搜索。
					\vskip 8pt
					\item 使用蒙特卡洛抽样的方法并结合估值函数对博弈树进行不完全展开。
				\end{itemize}
			\end{frame}
	
	\section{阶段性成果}
		\subsection*{基于手牌拆分的蒙特卡洛树搜索}
			\begin{frame}
				\frametitle{~~基于手牌拆分的MCTS搜索}
				\begin{itemize}
					\item 基于较小拆分和MCTS实现的“斗地主”游戏流程图
					\begin{figure}
						\centerline{\includegraphics<1->[scale=0.4]{figures/flow}}
					\end{figure}
				\end{itemize}
			\end{frame}
			
			\begin{frame}
				\frametitle{~~基于手牌拆分的MCTS搜索-实验结果}
				\begin{itemize}
					\item 与固定规则方法比较
					\begin{figure}
						\centerline{\includegraphics<1->[scale=0.5]{figures/result1}}
					\end{figure}
				\end{itemize}
			\end{frame}
			
			\begin{frame}
				\frametitle{~~基于手牌拆分的MCTS搜索-实验结果}
				\begin{itemize}
					\item 与固定规则方法比较
					\begin{figure}
						\centerline{\includegraphics<1->[scale=0.4]{figures/result2}}
					\end{figure}
				\end{itemize}
			\end{frame}
			
			\begin{frame}
				\frametitle{~~基于手牌拆分的MCTS搜索-实验结果}
				\begin{itemize}
					\item 与固定规则方法比较
					\begin{figure}
						\centerline{\includegraphics<1->[scale=0.4]{figures/result3}}
					\end{figure}
				\end{itemize}
			\end{frame}
			
			\begin{frame}
			\frametitle{~~基于手牌拆分的MCTS搜索-实验结果}
			\begin{itemize}
				\item 与7k7k小游戏世界斗地主智能Agent比较
				\begin{figure}
					\leftline{\includegraphics<1->[scale=0.35]{figures/result4}}
				\end{figure}
			\end{itemize}	
			
		\end{frame}
		
		\subsection*{CNN学习博弈状态收益}
			\begin{frame}
				\frametitle{~~CNN学习博弈状态收益}
				\begin{itemize}
					\item 训练损失变化如下:
					\begin{figure}
						\leftline{\includegraphics<1->[scale=0.18]{figures/cnn}}
					\end{figure}
				\end{itemize}
			\end{frame}
		
	\section{存在问题及后续安排}
		\subsection*{存在问题}
			\begin{frame}
				\renewcommand{\baselinestretch}{1.5}
				\frametitle{~~存在问题}
				\begin{itemize}
					\item 在对CNN学习结果进行分析中,发现针对某些情况的学习效果不佳
					\item 部分实验结果的分析不够全面
					\item 结果比较阶段,对于复现其他算法存在一定难度
				\end{itemize}
			\end{frame}
		\subsection*{后续工作安排}
			\begin{frame}
				\renewcommand{\baselinestretch}{1.5}
				\frametitle{~~后续工作安排}
				\begin{itemize}
					\item 解决存在的问题
					\item 整理实验数据、思路,明确论文框架,完善实验结果分析
					\item 根据进度安排,完成论文撰写及修改工作,完成毕业答辩等
				\end{itemize}
				
			\end{frame}
			\begin{frame}
				\renewcommand{\baselinestretch}{1.5}
				\frametitle{~~~后续工作安排}
				\begin{itemize}
					\item 进度安排如下:
				\end{itemize}
				\begin{table}[!htbp]
					\centering
					\label{jdap-table}
					\begin{tabular}{|c|c|l|}
						\hline
						\scriptsize 序号 & \scriptsize 时间 & \scriptsize 任务安排\\
						\hline
							1 & 2019.7-2020.02 & \tabincell{l}{\scriptsize 1.完善整个‘斗地主’过程,实现与人进行博弈\\
								\scriptsize  2.比较算法,整理数据\\
							\scriptsize 3.撰写论文\\						
							}\\
						\hline
						2 & 2020.2-2020.05 & \tabincell{l}{\scriptsize 1.修改论文\\
							\scriptsize 2.准备答辩\\			
						}\\
						\hline
					\end{tabular}
				\end{table}
			\end{frame}
			\begin{frame}
				\centering
				\zihao{2} 欢迎各位老师指导\\
				\zihao{1} 谢谢!
			\end{frame}	
\end{document}