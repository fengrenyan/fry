

\begin{abstract}
信息化的快速发展和深度应用所引发的隐私安全挑战,成为了制约数据开放、共享、交换和应用的瓶颈,并引起了法律界和学术界的高度关注。从技术的角度,差分隐私保护算法作为一种重要的隐私保护技术,在面向多维及其复杂关联的数据隐私保护方面的研究还不够成熟。首先,由于数据类型混合、稀疏性、域值空间大等原因,差分隐私的多维数据处理面临隐私脆弱性、计算效率低等方面的挑战;其次,数据融合的关联性、背景知识攻击和策略型敌手攻击,数据的隐私性和可用性的矛盾成为了突出问题。针对上述问题,从博弈的角度探讨隐私与效用的均衡及优化,不失为一种较好的解决方案。本文重点围绕隐私与效用均衡这一核心问题,基于信息熵、优化理论和博弈均衡等相关理论和方法,以构建均衡、优化模型为主线,在隐私量化方法设计、隐私与效用博弈模型构建及均衡求解、优化模型建立及求解等方面,取得了一系列的成果,为从技术和管理相结合的视角解决隐私保护问题提供了借鉴。本文所取得的主要研究成果包括:

1. 提出了差分隐私的信息熵度量模型及方法。针对隐私量化问题,基于Shannon基本通信模型,结合差分隐私的随机扰动原理,给出了有噪声信道的差分隐私通信模型及其形式化描述;进一步通过定义差分隐私保护模型中的信息熵、条件熵、联合熵、互信息量以及条件互信息量等概念,设计了以信息熵为核心的隐私度量模型;针对多维属性及其关联问题,基于图模型和马尔可夫模型等提出了面向多关联属性的差分隐私信息熵度量模型及方法,并基于数据处理不等式和Fano 不等式给出了信息泄露量的上下界。理论分析与实验结果表明,所提出的量化模型和方法能够有效地实现差分隐私量化目标,为隐私泄露风险评估和隐私保护机制设计提供了基础支撑。

2. 提出了含背景知识攻击的差分隐私优化模型。在所建立的差分隐私通信模型的基础上,结合所提出的隐私度量模型与方法和损失压缩理论,建立含背景知识的敌手模型,以此为基础提出了含背景知识攻击的差分隐私通信模型;在基于条件互信息量设计的隐私率失真函数基础上,提出了含背景知识攻击的最优化模型;进一步针对所提出优化模型的求解问题,利用Blahut-Arimoto交替最小化方法设计和实现了权衡隐私与效用的迭代最小化算法,并给出了其计算复杂度分析。
理论分析和实验仿真结果表明,所提出的相关方法相对于对称信道隐私保护机制具有明显优势。

3. 提出了有序随机响应扰动方案(Orderly Randomized Response Perturbation,ORRP)。针对多维数据差分隐私保护所面临的隐私脆弱性和效率低问题,面向数据收集场景的隐私保护需求,提出了一种有序随机响应扰动方案,有效解决了现有隐私保护机制忽视数据分布的影响、处理域值空间大和数据稀疏导致计算效率低的问题。该方案以所提出的隐私度量模型为基础,通过对隐私保护与数据可用性之间矛盾的分析和量化,给出了满足数据质量约束下最小化隐私泄露的互信息优化模型的形式化描述,以此实现最优隐私机制的概率密度函数计算并实现随机扰动。同时,参考独立并联信道模型,将上述结果推广到了多维数据情形。最后,从隐私泄露、数据可用性质量、关联度损失等方面给予了理论分析和实验仿真。结果表明,所提出的ORRP方案在数据语义完整性、隐私性和数据可用质量等方面比现有的结果更具有优势。

4. 提出了隐私保护的攻防博弈(Privacy-Preserving Attack Defense,PPAD)模型。针对差分隐私系统中存在消息灵通的策略型敌手问题,围绕数据收集场景设计了差分隐私保护的选择策略,以此为基础提出了隐私保护的攻防博弈模型,并通过均衡求解实现了差分隐私保护中隐私与效用之间的平衡。该方案基于所建立的差分隐私基本通信模型,在分析隐私保护方和策略型敌手方各自目标的基础上,构建了隐私极小极大优化模型,给出了包含参与者集合、策略空间、收益函数等的隐私攻防博弈模型的形式化描述。论文巧妙地利用了隐私互信息量的内涵与外延,构造了隐私保护的效用函数,最终具体实现了一个两方零和对策博弈模型的构建。论文利用极小极大定理和凹凸博弈理论给出了该攻防博弈模型的均衡分析,并进一步基于最优策略响应设计了均衡求解的策略优化选择算法。理论分析与数值实验结果表明,所提出的模型及方法能有效解决等价隐私保护机制之间的比较问题,同时可用于隐私保护程度最差情况下的隐私泄露风险评估。





\keywords{隐私度量,差分隐私保护,率失真函数,博弈均衡,优化模型}
\end{abstract}


\begin{englishabstract}
The challenges of privacy and security caused by the rapid development of informatization and in-depth applications, have become a bottleneck restricting data opening, sharing, exchange, and application, and have attracted great attention from the legal and academic communities. From the perspective of technology, the differential privacy (DP) protection algorithm, as an important privacy protection technology, is not mature enough in the research of data privacy protection for multi-dimensional and complex associations. Firstly, due to the mixed data types, sparseness, and large domain value space etc, the multi-dimensional data processing of DP is faced with the challenges such as privacy vulnerability and low computational efficiency. Secondly, the relevance of data fusion, background knowledge attacks and strategic adversary attacks, and the contradiction between data privacy and usability have become prominent issues. For the problems mentioned above, it is a better solution to investigate the trade-off and optimization of privacy and utility from the perspective of the game theory. Thus, this article mainly focuses on the crucial problem of the trade-off between privacy and utility. Based on information entropy, optimization theory and game equilibrium and other related theories and methods, the equilibrium and optimization models are constructed as the main line of this research. A series of results have been achieved in designing of privacy quantification methods, constructing and solving game model between privacy and utility, optimization model establishment and solving, etc., which provide a reference for solving privacy protection issues from the perspective of combining technology and management. The major contributions can be summarized as follows.

1. The information entropy metric models and methods of DP are proposed. For the quantitative problem of privacy, the noisy DP communication model and formalization statement are defined based on the Shannon's fundamental communication model and the randomized perturbation principle of DP. Further, the notions of information entropy, conditional entropy, joint entropy, mutual information and conditional mutual information, etc., are defined under the differential privacy model, and then, the privacy metric models with information entropy as the core are designed. For the problem of multi-dimensional and correlated attributes, based on the graph and Markov model, etc., a privacy metric model and method for multi-dimensional and correlated attributes is proposed. Then, the upper and lower bounds of privacy leakage are quantified by using data processing inequality and Fano's inequality. Theoretic analysis and experimental results are demonstrating the proposed metric model and method can effectively achieve the goal of DP quantification, and further provide basic support for privacy leakage risk assessment and privacy protection mechanism design.


2. The differential privacy optimization model with background knowledge attacks is proposed. Based on the established fundamental communication model of the DP, lossy compression theory and the proposed privacy metric model, the adversary model which has relevant background knowledge is established, and further the DP communication model with background knowledge attacks is proposed. By using conditional mutual information measures privacy, this paper updates the form of the well-known rate distortion function, and proposes the differential privacy optimization model with background knowledge attacks. Further, the alternating minimization iteration algorithm solving the proposed optimization model is designed and implemented based on the Blahut-Arimoto alternating minimization method, and the computation complexity analysis is provided. Theoretic analysis and experimental results are demonstrating the proposed method have significant advantages in data quality and privacy leakage when compared with the existing symmetrical channel mechanism.


3. The orderly randomized response perturbation (ORRP) scheme is proposed. For the problem of low efficiency and privacy vulnerability when deal with multi-dimensional data using local differential privacy, and facing the privacy protection requirements of data collection scenarios, this paper proposes an orderly randomized response perturbation scheme. The proposed ORRP scheme effectively solves the impact of the existing privacy protection mechanisms ignoring data distribution, and the problem of low computing efficiency caused by the large processing domain value space and sparse data. To be specific, the proposed ORRP scheme based on the prior proposed privacy metric model. A mutual information optimization model subjects to a given data quality loss constraint to minimize privacy leakage, is proposed by analyzing and quantifying the requirements of privacy and data quality. Further, the probability density function (PDF) of the optimal privacy mechanism is computed by the means above, and it is used to achieve randomized perturbation. Meanwhile, referring to the independent parallel channel model, the above methods are extended to the case of multi-dimensional data. Finally, theoretical analysis and experimental simulations are given in terms of privacy leakage, data usability quality, and correlation loss. The results demonstrate that the proposed ORRP has more advantages than the existing methods in terms of data semantic integrity, privacy and data availability quality.


4. The privacy-preserving attack and defense (PPAD) game model is proposed. For the problem of informed and strategic adversary in the differential privacy system, the selection strategy of differential privacy protection is designed around the data collection scenarios. On the basis of the above, the PPAD game model is proposed, and the trade-off between privacy and utility in the protection of differential privacy is achieved by solving the equilibrium. The proposed scheme is based on the established differential privacy basic communication model. The privacy minimax optimization model is established by analyzing the privacy goals of defender and strategic attacker, and further the formalization statement of PPAD is provided, which includes players' sets, strategic spaces and payoff functions etc. This paper cleverly uses the connotation and extension of private mutual information to construct the utility function of privacy protection, and finally realized the construction of a two-person zero-sum (TPZS) game model. Then, this paper provides the game analysis by using von Neumann's minimax theorem and concave-convex game, and further designs a strategy optimization selection algorithm to calculate saddle point based on the optimal strategy response. Theoretic analysis and numeric simulation results show that the proposed model and method can effectively solve the problem of comparison between equivalent privacy mechanisms, and also can be used for privacy leakage risk assessment in the worst case of privacy protection.







\englishkeywords{Privacy metric, differential privacy protection, rate-distortion function, game equilibrium, optimization model
}
\end{englishabstract}
