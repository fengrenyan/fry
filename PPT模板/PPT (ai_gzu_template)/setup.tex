%%%%%%%%%%%%%%%%%%%%%%%%%%%%%%%%%%%%%%%%%%%%%%%%%%%
% package
%%%%%%%%%%%%%%%%%%%%%%%%%%%%%%%%%%%%%%%%%%%%%%%%%%%
\usepackage{amsmath}

%\usepackage{beamerthemesplit}
\usepackage{graphicx}
\usepackage{ulem}
\usepackage{enumerate}
%\usepackage{hyperref}
\usepackage[english]{babel}
\usepackage[ruled,vlined]{algorithm2e}
%\usepackage{algorithmic}

\usepackage{xspace} %弹性space,针对各种newcommand的字体变形以后的空格问题。

\usepackage{ulem} %给文字添加下划线、波浪线等样式

\usepackage{upgreek} % \uptau 等数学符号


%\usepackage{ctex} %处理中文

%设置ref的颜色, documentclass[xcolor=dvipsnames]{beamer}
\usepackage{hyperref}
\hypersetup{
    colorlinks=true,
    linkcolor=Tan,
    filecolor=blue,      
    urlcolor=blue,
    citecolor=cyan,
}
% xlongleftarrow
\usepackage{extarrows}

% 输入代码
\usepackage{listings}

\definecolor{codegreen}{rgb}{0,0.6,0}
\definecolor{codegray}{rgb}{0.5,0.5,0.5}
\definecolor{codepurple}{rgb}{0.58,0,0.82}
\definecolor{backcolour}{rgb}{0.95,0.95,0.92}

\lstdefinestyle{mystyle}{
    backgroundcolor=\color{backcolour},   
    commentstyle=\color{codegreen},
    keywordstyle=\color{magenta},
    numberstyle=\tiny\color{codegray},
    stringstyle=\color{codepurple},
    basicstyle=\ttfamily\footnotesize,
    breakatwhitespace=false,         
    breaklines=true,                 
    captionpos=b,                    
    keepspaces=true,                 
    numbers=left,                    
    numbersep=5pt,                  
    showspaces=false,                
    showstringspaces=false,
    showtabs=false,                  
    tabsize=2
}
\lstset{style=mystyle}
\lstset{morekeywords={boolean,for,all,if,then,else, do, while, begin, and,or,choose,return,end,pairs,case}}

%绘图
\usepackage{tikz}



%%%%%%%%%%%%%%%%%%%%%%%%%%%%%%%%%%%%%%%%%%%%%%%%%%%


\mode<presentation>
{
  %设置主题
  %\usetheme{Singapore} % for print 
  \usetheme{Warsaw}
  %\usetheme{Copenhagen}
  
  %设置颜色主题,beaver,spruce
  %\usecolortheme{orchid}
  
  %设置背景
  \setbeamertemplate{background canvas}[vertical shading][bottom=teal!5,top=blue!5]
  
  %设置block特征
  %\setbeamertemplate{blocks}[rounded][shadow=true]
  
  %设置footline显示页码,但是可能会覆盖掉主题footline
  %\setbeamertemplate{footline}[frame number]
  
  %添加页码代码
  \expandafter\def\expandafter\insertshorttitle\expandafter{%
  \insertshorttitle\hfill%
  \insertframenumber\,/\,\inserttotalframenumber}
  
  \usefonttheme[onlysmall]{structurebold}
    
  % 设置数学公式的字体
  \usefonttheme[onlymath]{serif}
  % or ...
  %设置覆盖的效果,透明
  %\setbeamercovered{transparent}
  % or whatever (possibly just delete it)
  
  %\setbeamercolor{lemma in body}{bg=black}
  
  %定理等 计数
  \setbeamertemplate{theorems}[numbered] 
  
  % 设置beamer背景,第一个rule,改动第二个参数是高度,第二个rule,第一个参数是宽度
  \setbeamertemplate{background canvas}{
    \rule{0cm}{8cm}
    \rule{3.5cm}{0cm}
    \begin{tikzpicture}%
        \node  [opacity=0.08]{\includegraphics[width=5cm]{pics/gzu_logo_gray.png}};%
    \end{tikzpicture}
  }
  
}

% 在每一节前section有一张胶片显示目录
\AtBeginSection[]
{
\begin{frame}[allowframebreaks]
    %\setcounter{tocdepth}{1}
    %currentsection只显示当前章节,
    %\tableofcontents[currentsection,hideothersubsections,sectionstyle=show/hide,subsectionstyle=show/show/hide] 
    \tableofcontents[currentsection,subsectionstyle=show/show/hide]
\end{frame}
}

% 在每一子节前subsection有一张胶片显示目录
\AtBeginSubsection[]
    {
        \begin{frame}
        %\frametitle{#1}
        \tableofcontents[
        currentsection,
        currentsubsection,
        subsectionstyle=show/shaded/hide
        ]
        \end{frame}
    }

% 用于拆分内容多的theorem block到多个frame里面去
%\newcommand*{\theorembreak}{\usebeamertemplate{theorem end}\framebreak\usebeamertemplate{theorem begin}}


% for logo on the northeast corner.
% \usepackage{textpos}
% \addtobeamertemplate{headline}{}{%
% \begin{textblock*}{100mm}(.922\textwidth,-1.36cm)
% \includegraphics[height=1cm]{pics/gzu_logo_red.png}
% \end{textblock*}}




%%%%%%%%%%%%%%%%%%%%%%%%%%%%%%%%%%%%%%%%%%%%%%%%%%%%
% 设置自己的block
%%%%%%%%%%%%%%%%%%%%%%%%%%%%%%%%%%%%%%%%%%%%%%%%%%%%
% my Lemma block
%一些LATEX内部命令含有@字符,如\@addtoreset,
%如果需要在文档中使用这些内%部命令,就需要借助于
%另两个命令\makeatletter和\makeatother.

\makeatletter
\def\th@orangeblock{%
\normalfont % body font
\setbeamercolor{block title example}{bg=orange,fg=white}
\setbeamercolor{block body example}{bg=orange!20,fg=black}
\setbeamercolor{example text}{fg=red}
\def\inserttheoremblockenv{exampleblock}
}
\makeatother
\theoremstyle{orangeblock}
\newtheorem{mylemma}[theorem]{Lemma}%

% my theorem block
\makeatletter
\def\th@orchidblock{%
\normalfont % body font
\setbeamercolor{block title example}{bg=Orchid,fg=white}
\setbeamercolor{block body example}{bg=Orchid!20,fg=black}
\setbeamercolor{example text}{fg=blue}
\def\inserttheoremblockenv{exampleblock}
}
\makeatother
\theoremstyle{orchidblock}
\newtheorem{mythm}[theorem]{Theorem}%

%%%%%%%%%%%%%%%%%%%%%%%%%%%%%%%%%%%%%%%%%%%%%%%%%%%


%%%%%%%%%%%%%%%%%%%%%%%%%%%%%%%%%%%%%%%%
% CCS
%%%%%%%%%%%%%%%%%%%%%%%%%%%%%%%%%%%%%%%%
\newcommand{\act}{$Act$} % Actions

%%%%%%%%%%%%%%%%%%%%%%%%%%%%%%%%%%%%%%%%

\newcommand{\var}{\varphi}

\newcommand{\D}{\:\: |\:\:}

%孙子节
\newcommand{\ttl}{\subsubsection*}
%右箭头
\newcommand{\rto}{\rightarrow}
%左箭头
\newcommand{\lt}{\leftarrow}
\newcommand{\lto}{\leftarrow}
%左右箭头
\newcommand{\lrto}{\leftrightarrow}
%粗右箭头
\newcommand{\Rto}{\Rightarrow}
%粗左箭头
\newcommand{\Lto}{\Leftarrow}
%粗左右箭头
\newcommand{\LRto}{\Leftrightarrow}

%
\newenvironment{lprule}{\begin{center}\tt\begin{tabular}{rl}}{\end{tabular}\end{center}}
%
\newenvironment{smallenu}{\begin{enumerate}\setlength{\itemsep}{0.0pt}}{\end{enumerate}}
%
\newenvironment{smallitem}{\begin{itemize}\setlength{\itemsep}{0.0pt}}{\end{itemize}}

\long\def\comment#1{}

%not
\newcommand{\Not}{not \,}

\newcommand{\aligna}{\hspace{.1in}&}
\newcommand{\tab}{\hspace*{.2in}}

\newcommand{\is}{\stackrel{\triangle}{=}}

%粗体if
\newcommand{\mif}{{\bf if}}
%粗体while
\newcommand{\mwhile}{{\bf while}}
%粗体then
\newcommand{\mthen}{{\bf then}}
%粗体else
\newcommand{\melse}{{\bf else}}
%粗体do
\newcommand{\mdo}{{\bf do}}
%\newcommand{\mand}{\!\circ\!}
% &符号,带空格
\newcommand{\mand}{\;\mbox{\tt\&}\;}
%粗体!, 带空格
\newcommand{\cut}{\mbox{\tt !}}
%
\newcommand{\less}{\prec}
%
\newcommand{\sa}{same\mbox{-}actions}
%打字机字体
\newcommand{\firstproof}{\mbox{\tt first-der}}
%打字机字体
\newcommand{\minimal}{\mbox{\tt minimal}}
%%%%%%%%%%%%%%%%%%%%%%%%%%%%%%%%%
%特殊表达
\newcommand{\Holds}{H}
\newcommand{\holds}{H}
\newcommand{\Poss}{Poss}
\newcommand{\acc}{Acc}
\newcommand{\caused}{Caused}
\newcommand{\effect}{E\!f\!\!f\!ect}
\newcommand{\fluent}{F\!luent}
\newcommand{\complex}{Complex}
\newcommand{\defined}{De\!f\!ined}
\newcommand{\static}{Static}
\newcommand{\axiom}{Axiom}
\newcommand{\causes}{Causes}
\newcommand{\poss}{Precond}
\newcommand{\domain}{Domain}
\newcommand{\Con}{Const}
\newcommand{\Loop}{Loop}
\newcommand{\ES}{ES}
\newcommand{\vES}{es}
\newcommand{\IU}{IU}
\newcommand{\LF}{LF}
\newcommand{\comp}{C\!O\!M\!P}
\newcommand{\DLF}{D\!L\!F}
%%%%%%%%%%%%%%%%%%%%%%%%%%%%%%%%%








%%%%%%%%%%%%%%%%%%%%%%%%%%%%%%%%%
%功能键
%%%%%%%%%%%%%%%%%%%%%%%%%%%%%%%%%
\newcommand{\tabx}{\tab}
\newcommand{\tabxx}{\tab\tab}
\newcommand{\tabxxx}{\tabxx\tab}
\newcommand{\tabxxxx}{\tabxx\tabxx}
\newcommand{\atom}{\emph{$\mathcal{A}$toms}}
\newcommand{\ninf}{|\hspace{-.2cm}\sim}
\newcommand{\nninf}{|\hspace{-.2cm}\sim\hspace{-.42cm}/}
%?
\newcommand{\kmn}[3]{\mbox{$#1$-$#2$-$#3$}}
%%%%%%%%%%%%%%%%%%%%%%%%%%%%%%%%%

%斜体
\newcommand{\ground}{\textit{ground}}
%斜体
\newcommand{\Th}{\textit{Th}}
%斜体
\newcommand{\HB}{\textit{HB}}
%斜体
\newcommand{\wLF}{\textit{wLF}}
%斜体
\newcommand{\sLF}{\textit{sLF}}
%斜体
\newcommand{\dLF}{\textit{dLF}}
%斜体
\newcommand{\cLF}{\textit{cLF}}
%斜体
\newcommand{\nDF}{\textit{nDF}}
%斜体
\newcommand{\dIF}{\textit{dIF}}
%斜体
\newcommand{\cl}{\textit{cl}}
%斜体
\newcommand{\dnf}{\textit{dnf}}
%斜体
\newcommand{\defa}{\textit{def}}
%斜体
\newcommand{\Forget}{{\sf Forget}}
%斜体
\newcommand{\KForget}{\textit{KForget}}
%斜体
\newcommand{\GK}{\textit{GK}}
%斜体
\newcommand{\MKNF}{\textit{MKNF}}
%斜体
\newcommand{\MBNF}{\textit{MBNF}}
%斜体
\newcommand{\MMod}{\textit{MMod}}
%斜体
\newcommand{\BMod}{\textit{BMod}}
%斜体
\newcommand{\GMod}{\textit{GMod}}
%斜体
\newcommand{\GForget}{\textit{GForget}}
%斜体
\newcommand{\MForget}{\textit{MForget}}
%斜体
\newcommand{\BForget}{\textit{BForget}}
%斜体
\newcommand{\WFS}{\textit{WFS}}
%斜体
\newcommand{\Pred}{\textit{Pred}}
%斜体
\newcommand{\wunfold}{\textit{wunfold}}
%斜体
\newcommand{\unfold}{\textit{unfold}}
%斜体
\newcommand{\DL}{\textit{DL}}
%斜体
\newcommand{\Pos}{\textit{Pos}}
%斜体
\newcommand{\Atoms}{\textit{Atoms}}
%斜体
\newcommand{\Neg}{\textit{Neg}}
%斜体
\newcommand{\Head}{\textit{Head}}
%斜体
\newcommand{\Body}{\textit{Body}}
%斜体
\newcommand{\Var}{\textit{Var}}
%斜体
\newcommand{\CAN}{\textit{cano}}
%斜体
\newcommand{\plit}{\textit{plit}}
%斜体
\newcommand{\vLF}{\textit{lf}}
%斜体
\newcommand{\vcomp}{\textit{comp}}
%斜体
\newcommand{\gcomp}{\textit{Comp}}
%斜体
\newcommand{\BLoop}[1]{\textit{BLoop}($#1$)}
%斜体
\newcommand{\lfp}{\textit{lfp}}
%斜体
\newcommand{\gfp}{\textit{gfp}}
%斜体
\newcommand{\NES}{\textit{NES}}
%斜体
\newcommand{\ins}{\textit{Ins}}
%斜体
\newcommand{\gr}{\textit{ground}}
%伪造?\Vdash
\newcommand{\falsify}{\mbox{$\,\parallel\!\!\!-\,$}}
%\not\Vdash
\newcommand{\nfalsify}{\mbox{$\,\parallel\!\!\!/\!\!\!-\,$}}

%罗马字体
\newcommand{\ASforget}{\textrm{Forget}_{ST}}
%中等权重
\newcommand{\SM}{\textmd{SM}}

% 等线体
\newcommand{\forget}{{\sf Forget}}
% 等线体
\newcommand{\EWForget}{{\sf EWForget}}
% 等线体
\newcommand{\HT}{{\sf ht}}
% 等线体
\newcommand{\EQ}{{\sf eq}}
% 等线体
\newcommand{\Reduct}{{\sf Reduct}}
% 等线体
\newcommand{\SForgetLP}{{\sf SForget}}
% 等线体
\newcommand{\WForgetLP}{{\sf WForget}}
% 等线体
\newcommand{\IR}{{\sf IR}}

% 等线体
\newcommand{\Mod}{{\sf Mod}}
\newcommand{\END}{\textsf{END}\xspace}


%特殊符号?
\newcommand{\DNot}{\sim}
\newcommand{\NMdash}{\mbox{$|\!\!\!\thicksim$}}
\newcommand{\uminus}{\mbox{$\cup\!\!\!\textit{-}\,$}}
\newcommand{\cminus}{\mbox{$\cap\!\!\!\textit{-}\,$}}
\newcommand{\HTmodels}{\models}
\newcommand{\vMod}{|\!\!\models}


%命题
\newtheorem{proposition}{Proposition}


% 使用高亮盒子,dk mybox等
\usepackage{tcolorbox}
  \tcbuselibrary{listings,skins,breakable,xparse}
% 高亮盒子, 注意:不能分行! dk
\newtcbox{\dk}[1][orange!70!red]{on line,before upper={\rule[-0.2ex]{0pt}{1ex}\ttfamily},
  arc=0.8ex,colback=#1!30!white,colframe=#1!50!black,
  boxsep=0pt,left=1.5pt,right=1.5pt,top=1pt,bottom=1pt,
  boxrule=1pt}
  
% 高亮盒子,注意:不能分行! ntb note text box  
\newtcbox{\ntb}[1][ForestGreen!70!red]{on line,before upper={\rule[-0.2ex]{0pt}{1ex}\ttfamily},
  arc=0.8ex,colback=#1!30!white,colframe=#1!50!black,
  boxsep=0pt,left=1.5pt,right=1.5pt,top=1pt,bottom=1pt,
  boxrule=1pt}



% add logo to the slides
%\logo{\includegraphics[height=1cm]{pics/gzu_logo_red.png}}
% \usepackage{tikz}
% \logo{
% \begin{tikzpicture}%
%     \node  [opacity=0.08]{\includegraphics[width=5cm]{pics/gzu_logo_gray.png}};%
% \end{tikzpicture}
% }






