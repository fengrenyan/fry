% PPT名称
\title{CCS, PDL and Bisimularity}

% 作者名称
\author{Xin Zhou}
% %\vfill

% 研究机构
\institute{Guizhou University, China\\
Email:zhouxinj@hotmail.com
}

% PPT时间
\date{\today}%August 1st, 2012 @ Kunming, Yuannan}
%\date[ICLP'10]{In the 26th International Conference on Logic Programming}


% PPT 首页/标题页
\frame{\titlepage}
%\section*{Outline}
% PPT大纲
\begin{frame}{Outline}
\tableofcontents[hideallsubsections]
\end{frame}


%Title: Answer Set Programming: Theory and Practice
%Abstract:
%Answer set programming (ASP) is a declarative problem solving approach, initially tailored to
%modeling problems in the area of knowledge representation and reasoning (KRR). We will briefly
%introduce the basic theories of ASP and problem solving practice using ASP.

\section{Introduction}

\begin{frame}[allowframebreaks]{Introduction}
  \begin{itemize}
    \item \underline{CCS} Calculus of Communication systems is a \textcolor{teal}{process calculus} which introduced by Robin Milner in the 1980's\cite{DBLP:books/daglib/0067019}. It builds a general mathematical model, and skill in manipulating terms or expressions in oder to analyse the behaviour of these systems.
    

    \item \underline{PDL} is a formal system for reasoning about programs. Besides the traditional formalizing correctness specifications and proving their rigorously in a program.

    \item \underline{Forgetting} 
  \end{itemize}
\end{frame}

%%%%%%%%%%%%%%%%%%%%%%%%%%%%%%%%%%%%%%%%%%%%%
\section{CCS}
%%%%%%%%%%%%%%%%%%%%%%%%%%%%%%%%%%%%%%%%%%%%%



%%%%%%%%%%%%%%%%%%%%%%%%%%%%%%%%%%%%%%%%%%%%%%%%%%%%%%%
\section{PDL and Bisimulation}
%%%%%%%%%%%%%%%%%%%%%%%%%%%%%%%%%%%%%%%%%%%%%%%%%%%%%%%

 \subsection{Motivation and Introduction}

\begin{frame}{Motivation and introduction}

\underline{Propositional Dynamic Logic PDL} is a formal system for reasoning about programs. The most common operators are: non-deterministic choice $(\cup)$, sequential composition $(;)$, iteration $(\ast)$ and test $(?)$.

\begin{itemize}
    \item This logic's semantics is given by Labeled Transition Systems, where $R_\pi$ stand for a binary relation for each Program $\pi$.
    
    \item PDL can be used to prove that two programs $\pi_1$ and $\pi_2$ are logically equivalent $\models \langle \pi_1\rangle p \leftrightarrow \langle \pi_2\rangle p$ (where $\langle \pi_i \rangle p$ means that there is an execution of program $\pi_i$, such that after it, p holds). \cite{DBLP:journals/tcs/Benevides17}
    
    \item Try to show that there is the equivalence between bisimilar processes and logically equivalent programs in CCS.
\end{itemize}

\end{frame}
%  \input{chapters/PDL/sec1}
%  \input{chapters/PDL/sec2}
%  \input{chapters/PDL/sec3}
% \input{chapters/PDL/sec4}
%\input{chapters/PDL/sec5}















%%%%%%%%%%%%%%%%%%%%%%%%%%%%%%%%%%%%%%%%%%%%%%%%%%%%%%%%%%%%%%%%%%%%%
\section{Forgetting and Bisimulation}
%\subsection{F111}
%\subsection{F2}
%\subsection{F3}

%%%%%%%%%%%%%%%%%%%%%%%%%%%%%%%%%%%%%%%%%%%%%%%%%%%%%%%%%%%%%%%%%%%%%
%\section{Concluding Remarks and References}




%%%%%%%%%%%%%%%%%%%%%%%%%%%%%%%%%%%%%%%%%%%%%%%%%%%%%%%%%%%%%%%%%%%%%
% \begin{frame}
%     \begin{center}
%         \huge Thanks for your attention! \\Q \& A
%     \end{center}
% \end{frame}



%%%%%%%%%%%%%%%%%%%%%%%%%%%%%%%%%%%%%%%%%%%%%%%%%%%%%%%%%%%%%%%%%%%%%
% References
%%%%%%%%%%%%%%%%%%%%%%%%%%%%%%%%%%%%%%%%%%%%%%%%%%%%%%%%%%%%%%%%%%%%%
\begin{frame}[allowframebreaks] %allowramebreaks 可自动分页
        \frametitle{References}
        %\nocite{*}
        \bibliographystyle{alpha}
        %此文件虽然在chapters文件夹下,但是实际运行应该是在main.tex中,所以提取bib文件的路径要尤其注意。
        \bibliography{bib/ccs.bib}
\end{frame}

%%%%%%%%%%%%%%%%%%%%%%%%%%%%%%%%%%%%%%%%%%%%%%%%%%%%%%%%%%%%%%%%%%%%%
% End of the main content
%%%%%%%%%%%%%%%%%%%%%%%%%%%%%%%%%%%%%%%%%%%%%%%%%%%%%%%%%%%%%%%%%%%%%