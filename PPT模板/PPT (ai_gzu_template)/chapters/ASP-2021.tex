% PPT名称
\title{Answer Set Programming: Theory and Practice}

% 作者名称
\author{Yisong Wang}
% %\vfill

% 研究机构
\institute{Guizhou University, China\\
Email:ys\_wang168@sina.com
}

% PPT时间
\date{\today}%August 1st, 2012 @ Kunming, Yuannan}
%\date[ICLP'10]{In the 26th International Conference on Logic Programming}


% PPT 首页/标题页
\frame{\titlepage}
%\section*{Outline}
% PPT大纲
\frame{\tableofcontents}
%Title: Answer Set Programming: Theory and Practice
%Abstract:
%Answer set programming (ASP) is a declarative problem solving approach, initially tailored to
%modeling problems in the area of knowledge representation and reasoning (KRR). We will briefly
%introduce the basic theories of ASP and problem solving practice using ASP.

\section{Introduction}
\begin{frame}{Commonsense Knowledge/Reasoning: A challenge in AI}
  \begin{itemize}
    \item \underline{Common sense} is a basic ability to \textbf{perceive, understand}, and \textbf{judge} things that is shared by ("common to") nearly all people and can reasonably be expected of nearly all people without need for debate.

    \item \underline{Commonsense knowledge} is the collection of facts and information that an ordinary person is expected to know.

    \item \underline{Commonsense Reasoning} is concerned with simulating the human ability to make presumptions about the type and essence of ordinary situations they encounter every day.
  \end{itemize}
\end{frame}

\begin{frame}
 Progress in Commonsense:
  \begin{itemize}
   \item Commonsense has been identified as one of the most difficult and important problems in AI~\cite{Doyle:1984:AIM}.

    \item AI has seen great advances of many kinds
recently, but there is one critical area
where progress has been extremely slow:
ordinary commonsense~\cite{DBLP:journals/cacm/DavisM15}.
%ÈÕÇ° AI ÔÚÖî¶à·½Ã涼ÒÑÈ¡µÃ³¤×ã½ø²½£¬ µ«ÔÚÒ»¸ö¹Ø¼üÁìÓòµÄ½øÕ¹¼«Îª»ºÂý£º
% ÆÕͨ³£Ê¶(Ordinary Commonsense)
  \end{itemize}

  {Three well-known Commonsense Knowledge Formalisms:}
  \begin{itemize}
    \item Circumscription~\cite{DBLP:journals/ai/McCarthy80}.

    \item Default Logic~\cite{Reiter1980}.

    \item Nonmonotonic Logic I / Autoepistemic Logic~\cite{DBLP:journals/ai/McDermottD80,Moore:aij85}.

    \item Answer Set Programming~\cite{GelfondLifschitz88}.
  \end{itemize}


\end{frame}

\begin{frame}
  \frametitle{An Example: Graph Coloring}

  %\url{http://en.wikipedia.org/wiki/Graph_coloring#Definition_and_terminology}
  \begin{definition}
    Given a (directed) graph $G=(V,E)$ and a set $C$ of colors. Can we
    \underline{assign each node from $V$ a color from $C$} such that
    \underline{no two adjoint vertices have the same color}?
  \end{definition}

 \begin{tabular}{ll}
   % after \\: \hline or \cline{col1-col2} \cline{col3-col4} ...
   \begin{minipage}[h]{5.5cm}
  {\footnotesize It is expressed in first-order logic as:
   \begin{itemize}
     \item[(1)] $\forall X.  v(X)\supset \exists !Y .c(Y)\land clr(X,Y)$,
     \item[(2)] $\forall X,Y,C. e(X,Y)\supset\neg (clr(X,C)\land clr(Y,C))$ where
   %\end{itemize}
   %where
   %\begin{itemize}
     \item[] $v(X)$: $X$ is a node; \\
      $c(Y)$: $Y$ is a color;\\
      $e(X,Y)$: there is an edge between nodes $X$ and $Y$;\\
      $clr(X,C)$: node $X$ is colored using color $C$.
   \end{itemize}}
  \end{minipage}
  &
  \begin{minipage}[h]{5cm}
  {\footnotesize
   \begin{itemize}
     \item The problem is NP-complete.
     \item \href{http://www.adaptivebox.net/CILib/code/gcpcodes_link.html}{\uwave{Various implementations}}:
     \begin{itemize}
     \item \href{http://www.cs.sunysb.edu/~algorith/files/vertex-coloring.shtml}{\uwave{The Stony Brook Algorithm Repository,}}
     \item \href{http://webdocs.cs.ualberta.ca/~joe/Coloring/index.html\#Graph.Colorers}{\uwave{Joseph Culberson}}.
     \end{itemize}
   \end{itemize}}
  \end{minipage}
 \end{tabular}
\end{frame}
%
%\begin{frame}
%  \frametitle{A prolog solution}
%\footnotesize{\begin{tabular}{l}
%    \% file name: coloring1.pl
%    \% coloring(+Graph,+Colors,-Coloring)\\
%coloring(g(Vs,Es),Colors,Coloring):-\\
%   \hspace{1cm} gener(Vs,Colors,Coloring),\\
%   \hspace{1cm} test(Es,Coloring).\\
%\% gener(+Vertices,+Colors,-Coloring)\\
%gener([],\_,[]).\\
%gener([V$|$Vs],Colors,[V-C$|$T]):-\\
%   \hspace{1cm} member(C,Colors), \% non-deterministic generator of colors\\
%   \hspace{1cm} gener(Vs,Colors,T).\\
%\% test(+Edges,+Coloring)\\
%test([],\_).\\
%test([e(V1,V2)$|$Es],Coloring):-\\
%   \hspace{1cm} member(V1-C1,Coloring), \% find color of vertex V1\\
%   \hspace{1cm} member(V2-C2,Coloring), \% find color of vertex V2\\
%   \hspace{1cm} C1$\setminus$=C2,                 \% test the difference of colors\\
%   \hspace{1cm} test(Es,Coloring).\\
%\end{tabular}}
%\end{frame}
%
%\begin{frame}
%  \begin{tabular}{rl}
%    \begin{minipage}[h]{4.5cm}
%     \begin{figure}
%      % Requires \usepackage{graphicx}
%      \includegraphics[width=4cm]{color2.pdf}\\
%      \caption{{\footnotesize Petersen graph}}
%     \end{figure}
%    \end{minipage}
%    &
%    \begin{minipage}[h]{7cm}
%      \begin{figure}
%        \includegraphics[width=6cm]{color-pl.pdf}\\
%      \end{figure}
%    \end{minipage}
%  \end{tabular}
%\end{frame}
%In SWI-prolog,
%coloring(g([1,2,3,4,5,6,7,8,9,10],[e(1,2),e(1,6),e(1,5),e(2,7),e(2,3),e(3,8),e(3,4),e(4,9),e(4,5),e(5,10),e(6,8),e(6,9),e(7,9),e(7,10),e(8,10)]),[g,r,b],Coloring).


% \begin{frame}
%   %\frametitle{Graph coloring: A declarative solution}
%   \small{  \begin{tabular}{l}
%     \%  A declarative solution\\
%     \% A logic program encoding in ASP: color.lp\\
%      c(g;r;b).\\
%      1 \{clr(X,I) : c(I)\} 1  :- v(X).\% {\color{blue}each node has exact one color}\\
%      \hspace{3cm}:- clr(X,I), clr(Y,I), e(X,Y), c(I). \\
%     \hspace{3cm} \% {\color{blue}adjacent nodes can not have the same color} \\
%     hide.\\
%     show clr(X,Y).
%     \end{tabular}}
%  \begin{tabular}{lr}
%   % after \\: \hline or \cline{col1-col2} \cline{col3-col4} ...
%   \begin{minipage}[h]{5cm}
%   \small{
%   \% The input graph: p10\\
%     e(X,Y)  :- e(Y,X), v(X;Y).\\
%   v(1..10). \\
%     e(1,2).
%     e(1,5).
%     e(1,6).
%     e(2,7).
%     e(2,3).
%     e(3,8).
%     e(3,4).
%     e(4,9).
%     e(4,5).
%     e(5,10).
%     e(6,8).
%     e(6,9).
%     e(7,9).
%     e(7,10).
%     e(8,10).}
%   \end{minipage}
%   &
%   \begin{minipage}[thb]{4.5cm}
%     \begin{figure}
%       % Requires \usepackage{graphicx}
%       \includegraphics[width=4.5cm]{color2.pdf}
%     \end{figure}
%   \end{minipage}
%  \end{tabular}
% \end{frame}

% \begin{frame}
%   \frametitle{Graph coloring: A running result}
%   \begin{tabular}{lr}
%   \begin{minipage}[h]{6cm}
%     \mbox{lparse color.lp p10  $|$ clasp.}
%       \begin{figure}
%       % Requires \usepackage{graphicx}
%       \includegraphics[width=6cm]{color3.pdf}
%       \end{figure}
%   \end{minipage}
%   &
%   \begin{minipage}[h]{5cm}
%       \begin{figure}
%       % Requires \usepackage{graphicx}
%       \includegraphics<2>[width=5cm]{color4.pdf}
%       \end{figure}
%   \end{minipage}
%   \end{tabular}
% \end{frame}



% \begin{frame}
%   \frametitle{Hamiltonian cycle: A running result}
%   \begin{tabular}{lr}
%   \begin{minipage}[h]{6cm}
%     \mbox{lparse hc1.lp p10y  $|$ clasp.}
%       \begin{figure}
%       % Requires \usepackage{graphicx}
%       \includegraphics[width=6cm]{hc2.jpg}
%       \end{figure}
%   \end{minipage}
%   &
%   \begin{minipage}[h]{5cm}
%       \begin{figure}
%       % Requires \usepackage{graphicx}
%       \includegraphics<2>[width=5cm]{hc1.jpg}
%       \end{figure}
%   \end{minipage}
%   \end{tabular}
% \end{frame}


%\begin{frame}{Sodoku}
%  {\footnotesize
%  Sudoku Puzzles \--- http://www.tellmehowto.net/sudoku/veryhardsudoku.html
%
%  Sudoku Wikipedia \--- https://en.wikipedia.org/wiki/Sudoku
%
%  http://www.hakank.org/answer\_set\_programming/sudoku.lp
%  }
%  \begin{tabular}{lr}
%   \begin{minipage}[h]{5cm}
%%    \mbox{lparse sudoku.lp $|$ clasp.}
%       \begin{figure}
%      % Requires \usepackage{graphicx}
%       \includegraphics[width=5cm]{sudoku-1.jpg}
%       \end{figure}
%   \end{minipage}
%   &
%   \begin{minipage}[h]{5cm}
%       \begin{figure}
%      % Requires \usepackage{graphicx}
%       \includegraphics<2>[width=5cm]{sudoku-2.jpg}
%       \end{figure}
%   \end{minipage}
%  \end{tabular}
%\end{frame}
%
%\begin{frame}
%  \begin{quote}
%  {\footnotesize
%  \begin{verbatim}
%    val(1..9).
%    border(1;4;7).
%
%% alldifferent boxes
%1 \{ x(X,Y,N) : val(X;Y):
%    X1<=X:X<=X1+2:Y1<=Y:Y<=Y1+2 \} 1 :- val(N), border(X1;Y1).
%
%% alldifferent rows, columns, values
%1 { x(X,Y,N) : val(N) } 1 :- val(X;Y).
%1 { x(X,Y,N) : val(X) } 1 :- val(N;Y).
%1 { x(X,Y,N) : val(Y) } 1 :- val(N;X).
%
%% alternative:
%% :- 2 { x(X,Y,N) : val(N) }, val(X;Y).
%% :- 2 { x(X,Y,N) : val(X) }, val(N;Y).
%% :- 2 { x(X,Y,N) : val(Y) }, val(N;X).
%  \end{verbatim}
%  }
%  \end{quote}
%
%\end{frame}

\begin{frame}
Answer set programming (ASP) is a kind of logic programming, which is
  \begin{itemize}
    \item founded on the stable model semantics \cite{GelfondLifschitz88};
    \item a paradigm for modeling and solving search problems \cite{Baral:aspbook};
    \item accepted as a knowledge representation tool \cite{Brewka:CACM:2011};
    \item accompanied with effective systems: smodels, clasp, DLV;
    \item successfully applied in artificial intelligence and beyond:
    \pause
    \begin{itemize}
      \item product configuration, planning \cite{Lifschitz2002},
      \item reasoning about action, diagnosis \cite{Nogueiraetal2001};
      \item bio-informatics \cite{Gebser:TPLP:2011}, linguistics;
    \end{itemize}
  \end{itemize}
\end{frame}
%
%\begin{frame}
%  The ASP research communities:
%  \begin{itemize}
%    \item America \& Canada: University of Texas, Kentucky, New Mexico State, Alberta, Simon Fraser, $\ldots$
%    \item Europe: Helsinki U. of Technology, U. of Potsdam, Vienna U. of Technology, $\ldots$
%    \item Australia: UWS, Griffith, UNSW, $\ldots$
%    \item China: HKUST, IOS CAS, USTC, SYSU, GZU, $\ldots$
%  \end{itemize}
%  The ASP distinguished researchers:
%  \begin{itemize}
%    \item Valadimir Lifschitz, Mirek Truszczynski, Tran Cao Son, James Delgrande, Jia-Huai You, $\ldots$
%    \item Thomas Eiter, Illka Nielema, Torsten Schaub, $\ldots$
%    \item Norman Foo, Yan Zhang, Kewen Wang, $\ldots$
%    \item Fangzhen Lin, Yi-Dong Shen, Mingyi Zhang, $\ldots$
%  \end{itemize}
%\end{frame}

\section{Theories of ASP}
\subsection{Syntax}
\begin{frame}[t]
We assume a underlying propositional language $\cal L$.

A {\em (normal) rule} $r$ is an expression of the form
\begin{equation}\label{rule}
  A\lto A_1,\ldots, A_m,\Not A_{m+1},\ldots, \Not A_n
\end{equation}
where $A,A_i(1\leq i\leq n)$ are atoms. The rule (\ref{rule}) is often written as
\begin{equation}\label{rule:set}
  \Head(r)\lto \Pos(r),\Not\Neg(r),
\end{equation}
where $\Head(r)=A, \Pos(r)=\{A_1,\ldots,A_m\}$, $\Neg(r) = \{A_{m+1},\ldots, A_n\}$.

The rule $r$ is {\em Horn} if $m=n$.

\begin{itemize}
  \item A {\em normal logic program} is a finite set of rules.

  \item A {\em Horn logic program} is a  finite set of Horn rules.

  \item The {\em Herbrand base} of a program $P$ is the set of atoms occurring in $P$, denoted by $\HB_P$.
\end{itemize}
\end{frame}

\subsection{Answer set semantics}
\begin{frame}
An {\em interpretation} is a set of atoms.

Let $M$ be an interpretation, $A$ an atom, $r$ a rule of the form (\ref{rule:set}) and $P$ a program.
\begin{itemize}
  \item $M$ {\em satisfies} $A$ if $A\in M$;
  \item $M$ {\em satisfies} $\Not A$ if $M$ does not satisfies $A$;
  \item $A$ is extended to a set of literals of the form $A,\Not A$;
  \item $M$ {\em satisfies} $r$ if $M$ satisfies $\Pos\cup\Not\Neg$ implies $M$ satisfies $A$;
  \item $M$ {\em satisfies} $P$ if $M$ satisfies every rule of $P$;
\end{itemize}
we denote
\begin{itemize}
  \item satisfies: $\models$, and
  \item not satisfies: $\not\models$.
\end{itemize}

\end{frame}
%
%\begin{frame}
%\begin{definition}
%  Let $P$ be a Horn program and $M$ a set of atoms. The {\em immediate consequence operator} $T_P:2^{\HB_P}\rto 2^{\HB_P}$ is defined as
%  \[T_P(M)=\{A|(A\lto \Pos)\in P\ s.t\ M\models\Pos\}.\]
%\end{definition}
%
%Some properties of $T_P$:
%\begin{enumerate}
%  \item $T_P$ is {\em monotonic}, i.e. $M_1\subseteq M_2$ implies $T_P(M_1)\subseteq T_P(M_2)$.
%  \item The {\em least fix-point} of $T_P$ exists, denoted by $\lfp(T_P)$, and
%    \begin{itemize}
%      \item  $T_P^0=\emptyset$,
%      \item $T_P^{i+1}=T_P(T_P^i)$ for $i\ge 0$.
%      \item $\lfp(T_P)=T_P^\infty$.
%    \end{itemize}
%  \item $\lfp(T_P)$ is the least model of $P$.
%\end{enumerate}
%\end{frame}
%
%
%\begin{frame}
%Let $P$ be a logic program and $M$ a set of atoms.
%\begin{definition}[\cite{GelfondLifschitz88}]
% The GL-{\em reduct} of $P$ wrt $M$, written $P^M$, is
%obtained from $P$ by
%\begin{itemize}
%  \item deleting each rule of the form (\ref{rule:set}) if $\Neg\cap M\neq\emptyset$,
%  \item deleting all remaining $\Not B$.
%\end{itemize}
%A set $M$ is an {\em answer set} ({\em stable model}) of $P$ if $M=\lfp(T_{P^M})$.
%\end{definition}
%
% Let's consider the logic program $P=\{ a\lto \Not b;\quad       b\lto \Not a\}$.
%
% \begin{itemize}
%   \item $M=\{a\}$ is an answer set of $P$.% {\color{red}\textbf{why}?}
%
%%    \item Let $M_2=\emptyset$. We have that $P^{M_2}=\{a\lto; \quad b\lto\}$ whose least
%%    model if $\{a,b\}$ that is different from $M_2$. So that $M_2$ is not an answer set
%%    of $P$.
%   \pause
%   \item {\color{red}\textbf{Question}}: is $\emptyset$ an answer set of $P$?
%   \end{itemize}
%   \pause
%   \begin{theorem}
%   Deciding if a program $P$ has an answer set is NP-complete.
%   \end{theorem}
%\end{frame}
%
%
%\subsection{Well-founded semantics}
%\begin{frame}
%  \begin{itemize}
%  \item  A {\em literal} is of the form $A$ or $\Not A$ where $A$ is an atom.
%  \item The {\em complement} of a literal $l$, denoted by $\bar l$, is defined as
%  \begin{itemize}
%    \item $\overline A=\Not A$,
%    \item $\overline {\Not A}=A$.
%  \end{itemize}
%  \item A {\em partial interpretation} is a set of consistent literals.
%  \end{itemize}
%
%  Let $I$ be a partial interpretation, $l$ a literal and $T$ a set of literals.
%  \begin{itemize}
%    \item $I$ {\em partially satisfies} $l$ if $l\in I$,
%    \item $I$ {\em falsifies} $l$ if $\bar l\in I$,
%    \item $I$ {\em partially satisfies} $T$ if $I$ partially satisfies each literal in $T$,
%    \item $I$ {\em falsifies} $T$ if $I$ falsifies some literal in $T$.
%  \end{itemize}
%\end{frame}
%
%\begin{frame}
%  \begin{definition}[\cite{Gelderetal1991}]
%    Let $P$ be a logic program, $U$ a set of atoms and $I$ a partial interpretation.
%    $U$ is a {\em unfounded set} of $P$ wrt $I$ if, for every $p\in U$ and each rule $r\in P$ with
%    $\Head(r)=p$, either
%    \begin{enumerate}
%      \item $I$ falsifies $\Pos(r)\cup\Not\Neg(r)$, or
%      \item $\Pos(r)\cap U\neq\emptyset$.
%    \end{enumerate}
%  \end{definition}
%
%  Some properties:
%  \begin{itemize}
%    \item If $U_1$ and $U_2$ are unfounded sets of $P$ wrt $I$ then $U_1\cup U_2$ is a unfounded set of $P$ wrt $I$.
%    \item The greatest unfounded set exists for $P$ wrt $I$.
%    \item The greatest unfounded set for $P$ wrt $I$ can be evaluated in polynomial time.
%  \end{itemize}
%\end{frame}
%
%\begin{frame}
%\begin{definition}
%  Let $P$ be a logic program and $I$ a partial interpretation.
%  \begin{itemize}
%    \item ${\cal T}_P(I)$=$\{\Head(r)|$\mbox{$I$ partially satisfies $\Pos(r)\cup\Not\Neg(r)$}\}.
%    \item ${\cal U}_P(I)$ is the greatest unfounded set for $P$ wrt $I$.
%    \item ${\cal W}_P(I)={\cal T}_P(I)\cup\Not{\cal U}_P(I)$.
%  \end{itemize}
%  All the three operators are monotonic.
%
%  The {\em well-founded model} of $P$ is the least fixpoint of ${\cal W}_P$, $\lfp({\cal W}_P)$.
%\end{definition}
%
%For the logic program $P=\{a\lto\Not b; \quad b\lto \Not a\}$, the well-founded model is $\emptyset$.
%
%\pause
%{\color{red} Question}:
%\begin{itemize}
%  \item ${\cal T}_P(\emptyset)=?$
%  \item ${\cal U}_P(\emptyset)=?$
%\end{itemize}
%\end{frame}
%
%\begin{frame}
%  More properties: Let $P$ be a logic program.
%  \begin{enumerate}
%    \item The well-founded model of a logic program $P$ is polynomially computable.
%    \item $M$ is an answer set of $P$ iff $M\cup\Not\overline M$ is a fixpoint of ${\cal W}_P$.
%    \item If $p\in \lfp({\cal W}_P)$ then $p\in M$ for every answer set $M$ of $P$.
%    \item If $\Not p\in \lfp({\cal W}_P)$ then $p\notin M$ for every answer set $M$ of $P$.
%  \end{enumerate}
%
%  \begin{block}{Attention}
%  \alert{Well-founded semantics builds up the foundation of answer set evaluation.}
%  \end{block}
%\end{frame}
%
%\subsection{Algorithms for ASP}
%\begin{frame}
%  \frametitle{DPLL}
%  \begin{algorithm}[H]
%  \SetKwInOut{Input}{input}\SetKwInOut{Output}{output}
%  \Input{ A set of clauses $F$}
%  \Output{A Truth Value.}
%
%    \lIf{($F$ is a consistent set of literals)}{\Return true}\;
%    \lIf{($F$ contains an empty clause)}{\Return false}\;
%    \ForEach{(unit clause $l$ in $F$)}{$F \lto \textit{unit-propagate}(l, F)$}
%    \ForEach{(pure literal $l$ that occurs in $F$)}{$F \lto \textit{pure-literal-assign}(l, F)$}
%    $l \lto  \textit{choose-literal}(F)$\;
%    \Return{DPLL($F\land l$) or DPLL($F\land\neg l$)}
%    \caption{DPLL($F$)}
%  \end{algorithm}
%\end{frame}
%
%\begin{frame}
%  \frametitle{smodels \cite{Simonsetal2002}}
%  \begin{algorithm}[H]
%  \SetKwInOut{Input}{input}\SetKwInOut{Output}{output}
%  \Input{ A program $P$ and a set $A$ of literals}
%  \Output{A stable model.}
%
%  $A\lto \textit{expand}(P,A)$\;
%  \lIf{(conflict$(P,A)$)}{\Return false}\;
%  \lIf{($A$ covers $\textit{Atoms}(P)$)}{\Return $A^+$}\;
% % \lElseIf{($A$ covers $\textit{Atoms}(P)$)}{\Return $A^+$}\;
%%  \Else{
%%    $x\lto \textit{heuristic}(P,A)$\;
%%
%%    %\lIf{(smodels$(P,A\cup\{x\})$)}{\Return true}\;
%%    %\Else{\Return smodels$(P,A\cup\{\Not x\})$}
%%  }
%  $x\lto \textit{heuristic}(P,A)$\;
%  \Return{smodels$(P,A\cup\{x\})$ or smodels$(P,A\cup\{\Not x\})$}
%    \caption{smodels$(P,A)$}
%  \end{algorithm}
%
%  \vspace{.5cm}
%  \pause
%  {\color{red} Question}: Can we use SAT solvers for ASP?
%\end{frame}
%
%%\subsection{Loop formulas's approach}
%\begin{frame}
%  \frametitle{Completion}
% Let $P$ be a program and $h$ an atom in $\HB_P$.
% \begin{itemize}
% \item
%The {\em completion} of $h$ (relative to $P$), written
%$\comp(h,P)$, is the following formula:
%\begin{eqnarray*}
%  h\lrto \bigvee_{1\leq i\leq n} \left(\bigwedge_{A\in \Pos_i} A\wedge\bigwedge_{B\in\Neg_i}\neg B\right),
%\end{eqnarray*}
%where  $(h\lto \Pos_1,\Not\Neg_1),\ldots,(h\lto\Pos_n,\Not\Neg_n)$
%are all the rules in $P$ whose heads are the atom $h$;
%
%\item The {\em
%completion} of $P$, written $\comp(P)$, is the
%collection of completions of all atoms in $\HB_P$.
%\end{itemize}
%\end{frame}
%
%\begin{frame}
%  \frametitle{Loop formulas}
%  Let $P$ be a  program.
%  \begin{itemize}
%  \item The {\em positive dependency graph} of $P$, written $G_P$, is the
%directed graph $(\HB_P,E)$ where $(u,v)\in E$ if $P$ has a rule of the form (\ref{rule})
%s.t $A=u$ and $v=A_i$ for some $i~(1\le i\le m)$.
%
%  \item A nonempty subset $L$ of $\HB_P$ is a {\em loop} of $P$ if there is a cycle in $G_P$ which
%goes through only and all the nodes in $L$.
%  \item The {\em loop formula} of a loop $L$, written $\LF(L,P)$, is:
%\begin{eqnarray*}
%  \bigvee L\supset \bigvee_{1\leq i\leq n}\left(\bigwedge_{A\in\Pos_i}A\wedge\bigwedge_{B\in\Neg_i}\neg B\right)
%\end{eqnarray*}
%where $(h_1\lto\Pos_1,\Not\Neg_1),\ldots,
%(h_n\lto\Pos_n,\Not\Neg_n)$ are all the rules in $P$ such that
%$h_i\in L$ and $\Pos_i\cap L=\emptyset$.
%\end{itemize}
%\end{frame}
%
%\begin{frame}
%\begin{theorem}[\cite{LinZhao:assat}]
%  Let $P$ be a program. A set of atoms $M$ is an answer set of $P$ iff $M\models\comp(P)\cup\LF(P)$ where
%  $\LF(P)$ is the set of loop formulas of $P$.
%\end{theorem}
%
%For example,
%\begin{itemize}
%  \item Let $P_1=\{a\lto a; \quad a\lto\Not a\}$. $\comp(P_1)=\{a\lrto a\lor\neg a\}$, $L=\{a\}$ is the only loop of $P_1$,
%  $\LF(L,P_1)=a\supset\bot.$
%  $\comp(P_1)\cup\LF(P_1)$ is inconsistent. $P_1$ has no answer set.
%
%  \item Let $P_2=\{a\lto a\}$.
%
%  \pause
%  {\color{red}Question:} What's the model of $\comp(P_2)\cup \LF(P_2)$? %$\comp(P_2)=\{a\lrto a\}$,
%%  $$\LF(L,P_2)=a\supset\bot.$$
%%  The only model of $\comp(P_2)\cup \LF(P_2)$ is $\emptyset$.
%\end{itemize}
%\pause
%  \begin{block}{Good news: KR-06 best paper}The above theorem is extended for programs with variables \cite{CLWZ06}!
%  \end{block}
%\end{frame}

\subsection{ASP language and systems}
\begin{frame}
  The language of ASP systems:

  \begin{itemize}  {\color{blue}
    \item \href{www.tcs.hut.fi/Software/smodels/lparse.ps}{lparse 1.1},
    \item \href{http://potassco.sourceforge.net/}{gringo},
    \item \href{http://www.dlvsystem.com/dlvsystem/index.php/DLV}{dlv}.}
  \end{itemize}

  The ASP systems (for both windows and linux platforms):
  \begin{itemize}{\color{blue}
    \item \href{http://potassco.sourceforge.net/}{clasp serials}; % Germany; lparse and gringo
    \item \href{http://www.tcs.hut.fi/Software/smodels/}{smodels}; % Finland; lparse
    \item \href{http://www.dlvsystem.com/dlvsystem/index.php/DLV}{DLV};} %: Italy; dlv
    \item others: cmodels, assat, fasp, $\ldots$
  \end{itemize}
  \hrulefill
  \begin{itemize}{\color{blue}
    \item Tutorials of ASP: \href{http://potassco.sourceforge.net/teaching.html}{the website of clasp}
    \item The ASP competitions: \href{https://www.mat.unical.it/aspcomp2013}{1-4, 5,6,7}}
  \end{itemize}

\end{frame}

\begin{frame}
  \frametitle{ASP IDE}
  \begin{itemize}{\color{blue}
    \item \href{http://opus.bath.ac.uk/25392/}{APE}: Lparse
    \item \href{https://www.mat.unical.it/ricca/aspide/}{ASPIDE}: DLV
    \item \href{http://arxiv.org/abs/1109.3989}{SeaLion}: Gringo and DLV
    \item \href{http://igrom.sourceforge.net/index.php?url=page/index/user-guide}{iGROM}: DLV and Lparse
    \item \href{http://opus.bath.ac.uk/12628/}{ASPVIZ}
    \item \href{https://dtai.cs.kuleuven.be/krr/software/visualisation}{IDPDraw}}
    \item $\ldots$.
  \end{itemize}
\end{frame}
%
%\begin{frame}
%  \frametitle{lparse}
%  \begin{itemize}
%    \item Term: constant (lower case), variable (upper case), functions
%    \item Literal:
%    \begin{itemize}
%      \item basic literal: atom or its negation by default.
%      \item conditional literal: $p(X): q(X)$.
%      \item constraint literal: $lower \{l_1,\ldots,l_n\} upper$.
%      \item weight literal: $lower [l_1=w_1,\ldots,l_n=w_n] upper$.
%    \end{itemize}
%    \item rule types $A\lto \Body$:
%    \begin{itemize}
%      \item basic rule: $A$ is an atom;
%      \item choice rule: $A$ is a constraint literal or weight literal;
%      \item constraint rule: $A$ is $\bot$.
%    \end{itemize}
%  \end{itemize}
%\end{frame}
%
%\begin{frame}
%  Declaration statements:
%  \begin{itemize}
%    \item constant declaration: \#const n=10.
%    \item domain declaration: \#domain predicate(X).
%    \item external declaration: \#external pred(X).
%    \item functional declaration: \#function func\_name.
%    \item hide declaration: \#hide. or \#hide pred(X).
%    \item show declaration: \#show. or \#show pred(X).
%    \item weight declaration: \#weight literal = expr. or \#weight literal\_1=lieteral\_2.
%  \end{itemize}
%  Optimization statements:
%  \begin{itemize}
%    \item maximize:
%    \begin{itemize}
%      \item maximize $\{l_1,\ldots, l_n\}$.
%      \item maximize $[l_1=w_1,\ldots,l_n=w_n]$.
%    \end{itemize}
%    \item minimize:
%    \begin{itemize}
%      \item minimize $\{l_1,\ldots, l_n\}$.
%      \item minimize $[l_1=w_1,\ldots,l_n=w_n]$.
%    \end{itemize}
%  \end{itemize}
%\end{frame}
%
%\subsection{Extensions}
%\begin{frame}
%\frametitle{Disjunctive logic program}
%A {\em disjunctive rule} has the form
%\begin{equation}\label{dis:rule:set}
%  A\lto \Pos,\Not\Neg
%\end{equation}
%where $A$ is a set of atoms.
%
%\begin{itemize}
%\item  A {\em disjunctive logic program} $P$ is a finite set of disjunctive rules.
%
%\item The {\em Gelfond-Lifschitz reduction} is same as for normal logic programs.
%
%\item A set of atoms $M$ is an {\em answer set} of $P$ if $M$ is the least model of $P^M$.
%
%\item The problem of deciding if $P$ has an answer set is $\Sigma_2^P$-complete.
%\end{itemize}
%\end{frame}
%
%\begin{frame}
%  \frametitle{Extensions for atoms}
%  Aggregate programs \cite{DBLP:Faber:AIJ:2011}:
%  \begin{itemize}
%    \item A {\em ground set} is a pair $\langle \vec c, conj\rangle$ where $\vec c$ is a list of constants, and $conj$ a conjunction of atoms.
%    \item An aggregate function: $f(S)$ where $S$ a ground set and $f\in\{\#count, \#sum,\#min,\#max,\#times\}$.
%    \item An aggregate atom: $f(S)\circ T$ where $\circ\in\{=,\neq,>,\geqslant,<,\leqslant\}$ and $T$ a constant.
%    \item FLP-answer set semantics: $M$ is an answer set of $P$ if $M$ is the least model of $P_M$ where $P_M$ is the set of rules of $P$ whose bodies are satisfies by $M$.
%  \end{itemize}
%  \begin{block}
%    {There is no agreement on the semantics of aggregate programs.}
%  \end{block}
%\end{frame}
%
%\begin{frame}
%  Logic programs with constraint atoms \cite{Son:JAIR2007}: A constraint atom has the form $(D,C)$ where
%  \begin{itemize}
%    \item $D$ is a set of atoms, called {\em domain},
%    \item $C$ a subset of $2^D$, called {\em candidate solutions},
%    \item The {\em complement} of $(D,C)$ is $(D,2^D\setminus C)$,
%    \item A set of atoms $M$ {\em satisfies} $(D,C)$ if $M\cap D\in C$.
%  \end{itemize}
%
%   Let $S\subseteq M$ be two sets of atoms and $A=(D,C)$.
%   \begin{itemize}
%    \item $S$ {\em conditionally satisfies} $A$ wrt $M$, denoted by $S\models_MA$, if
%    \begin{itemize}
%      \item $S$ satisfies $A$, and
%      \item $I\in C$ for every $I$ with $S\cap D\subseteq I\subseteq M\cap I$.
%    \end{itemize}
%    \item Two answer set semantics: one is by reduct, the other is by complement.
%  \end{itemize}
%  \pause
%  \begin{block}
%    {We proposed a well-founded semantics for basic logic programs with constraint atoms \cite{Yisong:AAAI:2012}.}
%  \end{block}
%\end{frame}
%
%\begin{frame}
%Description logic programs \cite{DBLP:journals/ai/EiterILST08}: ${\cal K}=(O,P)$.
%
%A {\em dl-atom} is an expression of the form
%\begin{equation}\label{dl:atom}
%  \DL[S_1\ op_1\ p_1, \ldots,S_m\ op_m\ p_m;Q](\vec t),\ \  (m\ge 0)
%\end{equation}
%where
%\begin{itemize}
%  \item $op_i\in\{\uplus,\uminus,\cminus\}$;
%  \item each $S_i$ is a concept/role name if $op_i=\uplus$, otherwise it is
%   the negation of a  concept/role name;
%  \item $p_i$ is a unary predicate symbol if $S_i$ is a concept, and a binary predicate symbol
%   otherwise;
%  \item $Q(\vec t)$ is a {\em dl-query},
%    \begin{itemize}
%    \item $C(a)$ where $C$ is a concept, $\vec t=a$ and $a$ a constant;
%    \item $B\sqsubseteq C$, $\neg (B\sqsubseteq C)$, $R\sqsubseteq E$ or $\neg (R\sqsubseteq E)$;
%    \item$E(a,b)$ where  $E$ is a role, $\vec t=(a,b)$, $a,b$ are constants.
%    \end{itemize}
%\end{itemize}
%\end{frame}
%
%\begin{frame}
% An interpretation $I$ {\em satisfies} a dl-atom $A$ of the form (\ref{dl:atom}), denoted by $I\models_OA$, if $O(I;\lambda)\models Q(\vec t)$ where
% \begin{itemize}
%   \item    $O(I;\lambda)=O\cup\bigcup_{i=1}^{m}A_i(I)$\mbox{ and, for $1\leq i\leq m$,}
%   \item    \[A_i(I)=\left\{
%    \begin{array}{ll}
%        \{S_i(\vec e)\mid p_i(\vec e)\in I\}, & \hbox{if $op_i=\uplus$;} \\
%        \{\neg S_i(\vec e)\mid p_i(\vec e)\in I\}, & \hbox{if $op_i=\uminus$;} \\
%        \{\neg S_i(\vec e)\mid p_i(\vec e)\notin I\}, & \hbox{if $op_i=\cminus$;}
%    \end{array}
%    \right.
%    \]
%  \end{itemize}
%
%   A lots of answer set semantics:
%    \begin{itemize}
%      \item Stong/weak answer sets,
%      \item FLP-answer sets,
%      \item strongly/weakly well-supported answer sets \cite{Yi-Dong:IJCAI:2011}.
%    \end{itemize}
%
%
%  \pause
%  \begin{block}{What we did}
%  \begin{itemize}
%    \item the loop formulas characterization of dl-programs \cite{Yisong:TCS:2011},
%    \item \uwave{embedding dl-programs into default logic}.
%  \end{itemize}
%  \end{block}
%
%\end{frame}
%
%\begin{frame}
%  logic programs with evaluable functions \cite{LW:KR08}:
%  \begin{itemize}
%    \item A function of type $\tau_1\times\ldots\times\tau_n\rto \tau$ is {\em evaluable} if the domains $\tau_i~(1\le i\le n),\tau$ are
%  finite and fixed.
%    \item An interpretation is a set of atoms $\cal A$ associated with a function assignment $F$, $({\cal A},F)$.
%    \item The reduction is similarly defined, iteratively replace each function with its value.
%    \item Answer set semantics is given in a standard way.
%  \end{itemize}
%  \begin{block}
%    {One advantage is that many problems can be easily encoded!}
%    N-queens problem encoding:
%    \begin{align*}
%      &\lto q(x)=y,q(z)=y, x\neq z,\\
%      &\lto |q(x)-q(y)|=|x-y|,x\neq y.
%    \end{align*}
%  \end{block}
%\end{frame}
%
%\begin{frame}
%  \frametitle{First-order answer sets \cite{DBLP:AIJ:Ferraris:2011}}
%  Let $F$ be a first-order formula and $\textbf{p}$ a list of distinct predicate constants.
%  \[  \textrm{SM}_{\bf p}[F]=F\land \neg \exists \textbf{u}\left((\textbf{u}<{\bf p})\land F^*(\textbf{u})\right)\]
%  where
%  \begin{itemize}
%    \item $p\le q$: $\forall {\bf x}. p({\bf x})\supset q({\bf x})$; $p<q$: $(p\le q)\land \neg (q\le p)$.
%    \item $F^*({\bf u})$ is defined as
%    \begin{itemize}
%      \item ${\bf p(t)}^*={\bf u(t)}$, $q({\bf t})^*=q({\bf t})$ if $q\notin {\bf p}$;
%      \item $(F\circ G)^*=F^*\circ G^*$ where $\circ\in\{\land,\lor\}$;
%      \item $(F\rto G)^*=(F^*\rto G^*)\land (F\rto G)$;
%      \item $(QxF)^*=Qx(F^*)$ where $Q\in\{\forall,\exists\}$.
%    \end{itemize}
%  \end{itemize}
%
%  The {\em ${\bf p}$-stable models} of $F$ are the interpretations satisfying $\textrm{SM}_{\bf p}[F]$.
%\end{frame}

%\subsection{smodels}
%\subsection{clasp}
%\subsection{DLV}

\section{Applications of ASP}
\subsection{Graph domain}
\begin{frame}
  \frametitle{Hamiltonian circuit}
  \textbf{Problem}: Given a graph $G=(V,E)$, find a cycle $L$ of $G$ such that it goes through every nodes of $V$ exact once.

  ASP encode:
  \begin{center}
  \small{\begin{tabular}{rl}
  &  \% hc1.lp\\
  &\% {\color{blue} GENERATE }\\
  \{in(X,Y)\} & :- e(X,Y).\\
  &\% {\color{blue}TEST} \\
    &:- 2 \{in(X,Y) : e(X,Y)\}, v(X).\\
    &:- 2 \{in(X,Y) : e(X,Y)\}, v(Y).\\
    & \% {\color{blue} DEFINE: all nodes are REACHED}\\
    r(X) &:- in(1,X), v(X).\\
    r(Y) &:- r(X), in(X,Y), e(X,Y).\\
    &:- not r(X), v(X).\\
   & \#hide.\\
   & \#show in(X,Y).
  \end{tabular}}
  \end{center}
\end{frame}

% \begin{frame}
%  \begin{tabular}{lr}
%   % after \\: \hline or \cline{col1-col2} \cline{col3-col4} ...
%   \begin{minipage}[h]{5cm}
%   \small{
%   \% The input graph: p10y\\
%     e(X,Y)  :- e(Y,X), v(X;Y).\\
%   v(1..10). \\
%     e(1,2).
%     e(1,5).
%     e(1,6).
%     e(1,7).
%     e(2,7).
%     e(2,3).
%     e(3,8).
%     e(3,4).
%     e(4,9).
%     e(4,5).
%     e(5,10).
%     e(6,8).
%     e(6,9).
%     e(7,9).
%     e(7,10).
%     e(8,10).}
%   \end{minipage}
%   &
%   \begin{minipage}[h]{5cm}
%     \begin{figure}
%       % Requires \usepackage{graphicx}
%       \includegraphics[width=5cm]{hc1.pdf}
%     \end{figure}
%   \end{minipage}
%  \end{tabular}
%  %\mbox{lparse hc.lp p10y | clasp}
% \begin{figure}
%   % Requires \usepackage{graphicx}
%   \includegraphics[width=8cm]{hc2.pdf}
% \end{figure}

% \end{frame}
%
%\subsection{Planning domain}
%\begin{frame}
%  \frametitle{Blocks world \cite{Lifschitz2002}}
%  \begin{definition}
%    A set of blocks are on a table (initial state). The goal is to build one or more vertical stacks of blocks (goal state) such that
%    \begin{itemize}
%      \item each block can be on another block or on table,
%      \item a block can not be moved unless nothing on it.
%    \end{itemize}
%     %\textit{Its optimal solution (shortest plan) is NP-complete.}
%  \end{definition}
%  \begin{minipage}[h]{12cm}
%    \footnotesize{
%    \%block.lp\\
%  time(0..end).\\
%  loc(table).\\
%  loc(B) :- blk(B).\\
%  \% {\color{blue} \textbf{GENERATE}}\\
%  \{mv(B,L,T) : blk(B): loc(L)\}grippers :- time(T),T$<$end.\\
%  \% {\color{blue}\textbf{DISPLAY}}\\
%  \#hide.\\
%  \#show mv(B,L,T).
%}
%  \end{minipage}
%\end{frame}
%
%\begin{frame}
%  \begin{minipage}[h]{10cm}
%  \footnotesize{
%  \% {\color{blue}\textbf{DEFINE}}\\
%  \% effect of moving a block\\
%  on(B,L,T+1) :- mv(B,L,T),blk(B),loc(L),time(T),T$<$end.  \\\\
%  \% inertia\\
%  %\rule[3pt]{\textwidth}{0.5pt}\\
%  on(B,L,T+1) :- on(B,L,T),$\Not$-on(B,L,T+1),loc(L),blk(B),time(T),T$<$end.\\\\
%  \% uniqueness of location \\
%  -on(B,L1,T) :- on(B,L,T), L!=L1, blk(B),loc(L),loc(L1),time(T).\\\\
%    \% {\color{blue} \textbf{TEST}}\\
%  \% two blocks cannot be on top of the same block\\
%  :-2 \{on(B1,B,T):blk(B1)\}, blk(B), time(T).\\\\
%  \% a block can't be moved unless it is clear\\
%  :- mv(B,L,T), on(B1,B,T),blk(B),blk(B1),loc(L),time(T),T$<$end.\\\\
%  \% a block can't be moved onto a block that is being moved also\\
%  :- mv(B,B1,T),mv(B1,L,T),blk(B),blk(B1),loc(L),time(T),T$<$end.
%  }
%  \end{minipage}
%\end{frame}
%
%\begin{frame}
%\begin{tabular}{rl}
%\begin{minipage}[h]{3cm}
%\footnotesize{
%\%blk-1\\
%const grippers=2.\\
%const end=3.\\
%blk(1..6).\\
%\% {\color{blue} initial state}\\
%on(1,2,0).\\
%on(2,table,0).\\
%on(3,4,0).\\
%on(4,table,0).\\
%on(5,6,0).\\
%on(6,table,0).\\
%\% {\color{blue} goal state}\\
%:- not on(3,2,end).\\
%:- not on(2,1,end).\\
%:- not on(1,table,end).\\
%:- not on(6,5,end).\\
%:- not on(5,4,end).\\
%:- not on(4,table,end).}
%  \end{minipage}
%  &
%  \begin{minipage}[h]{7cm}
%\begin{figure}
%  % Requires \usepackage{graphicx}
%  \includegraphics[width=7cm]{block1.pdf}\\
%  \includegraphics[width=8cm]{block2.pdf}
%\end{figure}
%\end{minipage}
%  \end{tabular}
%\end{frame}
%
%\subsection{Bio-informatics}
%\begin{frame}
%  \frametitle{Metabolic network completion \cite{Torsten:ICLP:2009,DBLP:journals/ploscb/PrigentFDTLCGGE17}: A challenge}
%  \begin{itemize}
%    \item {\em Metabolic network}: a directed bipartite graph $G=(R\cup M,E)$,
%     \begin{itemize}
%       \item $R$: {\em reactions};
%       \item $M$: {\em metabolites};
%       \item $(m,r)\in E$: $m$ is a {\em reactant} of $r$, $reac(r)=\{m\in M|(m,r)\in E\}$;
%       \item $(r,m)\in E$: $m$ is a {\em product} of $r$, $prod(r)=\{m\in M|(r,m)\in E\}$.
%     \end{itemize}
%     \item {\em Scope of $M'\subseteq M$}, denoted by $\Sigma_G(M')$: the closure of $M'$ under reachability from $M'$,
%     \begin{itemize}
%       \item a reaction $r$ is {\em reachable} from $M'$ if $reac(r)\subseteq M'$;
%       \item a metabolite $m$ is {\em reachable} from $M'$ if either
%       \begin{itemize}
%         \item $m\in M'$, or
%         \item $m\in prod(r)$ for some $r$ being reachable from $M'$.
%       \end{itemize}
%     \end{itemize}
%  \end{itemize}
%\end{frame}
%%Metabolic networks describe the relationships between small biomolecules (metabolites) and the enzymes (proteins) that interact with them to catalyze a biochemical reaction.
%%
%%From: Handbook of Systems Biology, 2013
%
%%Metabolic Network
%%A metabolic network is the complete set of metabolic and physical processes that determine the physiological and biochemical properties of a cell. As such, these networks comprise the chemical reactions of metabolism, the metabolic pathways, as well as the regulatory interactions that guide these reactions.
%%Wikipedia  Wikipedia
%
%\begin{frame}
%\frametitle{Completion}
%  Given
%  \begin{enumerate}
%    \item a draft metabolic network $(R\cup M,E)$,
%    \item two sets $S,T\subseteq M$ ({\em seed} and {\em target}), and
%    \item a reference network  $(R'\cup M',E')$,
%  \end{enumerate}
%  $R''\subseteq R'\setminus R$ is the {\em completion} of
%  $(R\cup M,E)$ from $(R'\cup M',E')$ wrt $(S,T)$ if $T\subseteq \Sigma_G(S)$ where
%  \begin{itemize}
%    \item $G=((R\cup R'')\cup (M\cup M''),E\cup E'')$,
%    \item $M''=\{m\in M'|r\in R'', m\in reac(r)\cup prod(r)\}$, and
%    \item $E''=\{(m,r)\in E'|r\in R'', m\in reac(r)\}\cup \{(r,m)\in E'|r\in R'',m\in prod(r)\}$.
%  \end{itemize}
%\end{frame}
%
%\begin{frame}
%  \frametitle{The ASP encoding: I}
%  Meneco: \url{https://bioasp.github.io/apps.html}
%
%
%   Given draft and reference metabolic networks $G_d$ and $G_r$, $S, T\subseteq M$ (seed and target), \underline{${\cal C}(G_d,G_r,S,T)$}:
%   {\small\begin{align*}
%   & \underline{C(G_d)\cup C(G_r)\cup \{draft(d)\}\cup \{seed(s)|s\in S\}\cup \{target(t)|t\in T\}}.  \\
%   \\
%   &    \{xreaction(R):\Not reaction(R,N):draft(N)\}.
%   \end{align*}
%   \begin{align*}
%    xscope(M)&\lto seed(M).\\
%    xscope(M)&\lto product(M,R), reaction(R,N), draft(N), \\
%             &  xscope(M'):reactant(M',R).\\
%             & \mbox{\% \uwave{{\color{blue} all the rectants of R are in the scope}}}\\
%    xscope(M)&\lto product(M,R),xreaction(R), xscope(M'):reactant(M',R).\\
%                 &\lto target(M),\Not xscope(M).
%   \end{align*}}
%\end{frame}
%%
%%\begin{frame}
%%\frametitle{The ASP encoding: II}
%% {\small
%%  \begin{align*}
%%    &\lto xreaction(R), not ireaction(R).\\
%%    \\
%%    ireaction(R) &\lto interesting(M), product(M,R), reaction(R, N).\\
%%    interesting(M) &\lto target(M), not dscope(M).\\
%%    interesting(M) &\lto reactant(M,R), ireaction(R), not dscope(M). \\
%%    \\
%%    &\lto xreaction(R), not oreaction(R).\\
%%    oreaction(R) &\lto xscope(M) : reactant(M, R),reaction(R, N), not draft(N).
%%  \end{align*}
%%  }
%%\end{frame}
%
%\begin{frame}
%    Given a metabolic network $G_n=(R\cup M,E)$, the $C(G_n)$ includes:
%    \begin{itemize}
%      \item $\{reaction(r,n)|r\in R\}$,
%      \item $\{reactant(m,r)|m\in reac(r),r\in R\}$, and
%      \item $\{product(m,r)|m\in prod(r), r\in R\}$.
%    \end{itemize}
%  \begin{theorem}
%  $X$ is an answer set of the above ASP iff   $\{r|xreacton(r)\in X\}$ is a completion of $G_d$ from $G_r$ wrt $(S,T)$.
%  \end{theorem}
%  \pause
%  \begin{block}
%  {To minimize the number of reactions:}
%  \[minimize\{xreaction(R): \Not reaction(R,N): draft(N)\}.\]
%  \end{block}
%\end{frame}
%
%\begin{frame}
%  To refine the above encoding:
%  {\small \begin{align*}
%     &\mbox{\% {\color{blue}xreaction $R$ should be ``interesting".}}\\
%                    & \lto xreaction(R),\Not ireaction(R).\\
%     ireaction(R)   & \lto interesting(M), product(M,R), reaction(R,N).\\
%     interesting(M) & \lto target(M),\Not dscope(M).\\
%     interesting(M)&\lto reactant(M,R),ireaction(R), \Not dscope(M).\\
%       &\mbox{\% {\color{blue}xreaction $R$ should be ``operatable".}}\\
%                   &\lto xreaction(R), \Not oreaction(R).\\
%     oreaction(R)  &\lto xscope(M):reactant(M,R), reaction(R,N),\Not draft(N).\\
%     dscope(M) & \lto seed(M).\\
%     dscope(M) & \lto product(M,R), reaction(R,N), draft(N), \\
%     &dscope(M'):reactant(M',R).
%  \end{align*}}
%  {\color{red}\textbf{To run}}: gringo *.lp $|$ clasp  \% gringo is in version 2.0.2
%\end{frame}
%
%\begin{frame}
%  \frametitle{More applications}
%  See \href{http://www.kr.tuwien.ac.at/research/projects/WASP/report.html}{``WP5 Report: Model Applications and Proofs-of-Concept"}
%  \begin{itemize}
%    \item agent programming;
%    \item cryptanalysis;
%    \item diagnosis;
%    \item insurance industry;
%    \item model checking;
%    \item policies;
%    \item security protocols;
%    \item robot control;
%    \item $\ldots\ldots$.
%  \end{itemize}
%  asparagus:\url{https://asparagus.cs.uni-potsdam.de/instanceclass}
%\end{frame}
%
%\section{Some Challenges}
%\begin{frame}
%   \frametitle{Challenges in ASP \cite{Gebser:LPNMR:2011}}
%   ASP solving process:
%   \begin{enumerate}
%     \item Modeling: it is an art with craft, experience, and knowledge.
%     \item Grounding:
%      \begin{itemize}
%        \item  It leads to exponential blow up.
%        \item  It is a bottleneck of ASP solving.
%        \item  It leads to first-order ASP.
%      \end{itemize}
%     \item Solving;
%     \item \textit{Visualizing, Software Engineering.}
%   \end{enumerate}
%   Other aspects:
%   \begin{itemize}
%     \item Extensions, such as modulo theories.
%     \item \textsf{Standardization}
%     \item More applications
%     \item $\ldots$.
%   \end{itemize}
%
%\end{frame}

\section{Concluding Remarks and References}

\begin{frame}
\begin{itemize}
  \item Theories of ASP
  \begin{itemize}
    \item syntax,
    \item semantics,
    \item solving.
  \end{itemize}
  \pause
  \item Applications of ASP
  \begin{itemize}
    \item graph domain,
    \item planning,
    \item bio-informatics.
  \end{itemize}
  \pause
  \item Challenges in ASP
    \begin{itemize}
      \item grounding,
      \item engineering,
      \item $\ldots$.
    \end{itemize}
\end{itemize}
\end{frame}

\begin{frame}
 \begin{block}
    {\centering{\Huge Questions and Discussions?}}
  \end{block}
  \pause
  \vspace{2cm}
 \begin{block}
    {\centering{\Huge Thank you!}}
    \begin{center}
      {\large Thank you!}
    \end{center}
  \end{block}
\end{frame}

%  {\scriptsize
%  \bibliographystyle{alpha}
%  \bibliography{../bib/DL.bib}
%  }

\begin{frame}[allowframebreaks] %allowramebreaks 可自动分页
        \frametitle{References}
        %\nocite{*}
        \bibliographystyle{alpha}
        %此文件虽然在chapters文件夹下,但是实际运行应该是在main.tex中,所以提取bib文件的路径要尤其注意。
        \bibliography{bib/wys_DL.bib}
\end{frame}






